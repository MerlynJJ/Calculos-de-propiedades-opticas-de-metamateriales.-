\documentclass[12pt]{article}
\usepackage{graphicx}
\usepackage[spanish]{babel}
\usepackage{float}
\usepackage[utf8]{inputenc}
\usepackage[T1]{fontenc}
\usepackage{outline}
\usepackage{pmgraph}
\usepackage[normalem]{ulem}
\usepackage{amsmath}
\usepackage{amsfonts,amssymb,amsthm}
\usepackage{fancyhdr}
\usepackage{multicol}
\usepackage{vmargin}
\usepackage{tikz}
\usepackage{multicol}
\usepackage{bm}
\usepackage{hyperref}
\usepackage[scaled]{helvet}
%\renewcommand\familydefault{\sfdefault} 
%\usepackage{showkeys}
%\usepackage{showlabel}
\hypersetup{
    colorlinks=true,
    linkcolor=black,
    filecolor=magenta,
    urlcolor=blue,
}

\title{Metamateriales \\
  XVIII Escuela de Verano en Física}
\author{ Instituto de Ciencias Físicas \\
      \small Wolf Luis Mochán Backal }
\date{}
    

\begin{document}
\maketitle 
\tableofcontents

\section{Introducción}


Un meta-material es un material diseñado artificialmente con dos o más
materiales, con propiedades determinadas por la geometría de las
componentes y los periodos de disposición de los materiales en el
arreglo. Los meta-materiales se definen como ``{\em Un arreglo de
  elementos estructurales artificiales, diseñados para alcanzar
  propiedades electromagnéticas ventajosas e
  inusuales}''\cite{Metamorphose}, esta definición es la adoptada en
el {\em Virtual Institute for Artificial Electromagnetic Materials and
  Meta-materials}.

Análogo a los materiales usuales, las propiedades electromagnéticas de
meta-materiales son determinadas por sus elementos constituyentes
básicos, a los que se le denomina como meta-átomos, los cuales son
objetos hechos de materiales usuales.Las propiedades de los
meta-materiales pueden ser muy distintas a las de los materiales que
los conforman y pueden llegar a ser muy exóticas y pueden ser
especificadas escogiendo las formas, estructuras internas, tamaños,
orientaciones mutuas, etc., de sus meta-átomos.  Incluso sus repuestas
individuales pueden ser controladas por señales externas e internas y
por microprocesadores
programables.\cite{IntroductiontoMetamaterialsandNanophotonics}

Por ejemplo, si un meta-material tiene componentes metálicos, estos
pueden presentar resonancias debidas a un movimiento oscilatorio
colectivo de sus electrones de conducción, denominadas de acuerdo a
sus características como plasmones de bulto, plasmones de superficie o
plasmones localizados. La frecuencia de dichas oscilaciones se
denomina como frecuencia de plasma $\omega_{p}$. Para estimar esta
frecuencia, considere el modelo más simple de un conductor, un gas de
electrones de densidad de número $n_{0}$, moviendose libremente en un
entorno positivamente cargado. Si debido a algún efecto como una
compresión o rarefacción del gas de electrones se produce una
acumulación de carga $Q$, localizada en alguna región $\mathcal{R}$,
esta producirá un campo eléctrico de Coulomb $\vec{E} (\vec{r}) =
Q\vec{r}/r^{3} $ ilustrado en la figura \ref{Bulkplasmon}. De acuerdo
a la segunda ley de Newton, los electrones obtienen una aceleración
$d\frac{d^{2}\vec{r}}{dt^{2}} = -e\vec{E}(\vec{r},t)$, donde $m$ y
$-e$ son la masa y la carga eléctrica del electrón acelerado. La
aceleración de los electrones resulta en una corriente eléctrica dada
por $\frac{\partial \vec{j}(\vec{r},t)}{\partial t}=
\frac{n_{0}e^{2}}{m}\vec{E}(\vec{r},t)$, como $\vec{j}(\vec{r},t) =
-n_{0}e\frac{d\vec{r}}{dt}$. Integrando la ecuación diferencial de la
corriente sobre una superficie que rodee completamente la carga y
usando la ecuación de continuidad y la Ley de Gauss para el campo
eléctrico obtenemos una ecuación diferencial para la carga,
\begin{figure}[H]
\centering
\label{Bulkplasmon}


\tikzset{every picture/.style={line width=0.75pt}} %set default line width to 0.75pt        

\begin{tikzpicture}[x=0.75pt,y=0.75pt,yscale=-1,xscale=1]
%uncomment if require: \path (0,300); %set diagram left start at 0, and has height of 300

%Shape: Rectangle [id:dp5857108563848268] 
\draw [draw opacity=0][fill={rgb, 255:red, 198; green, 244; blue, 240
  } ,fill opacity=0.48 ] (148,66) -- (390.17,66) -- (390.17,269) --
(148,269) -- cycle ;
%Shape: Circle [id:dp3780902414948827] 
\draw [dash pattern={on 0.84pt off 2.51pt}] (217.04,166.52)
.. controls (217.04,135.56) and (242.14,110.46) .. (273.1,110.46)
.. controls (304.07,110.46) and (329.17,135.56) .. (329.17,166.52)
.. controls (329.17,197.48) and (304.07,222.58) .. (273.1,222.58)
.. controls (242.14,222.58) and (217.04,197.48) .. (217.04,166.52) --
cycle ;
%Shape: Polygon Curved [id:ds008835726195197124] 
\draw [fill={rgb, 255:red, 248; green, 231; blue, 28 } ,fill opacity=1
] (277.08,151.5) .. controls (296.08,166.5) and (293.08,169.5)
.. (269.08,167.5) .. controls (248.08,193.5) and (254.08,143.5)
.. (277.08,151.5) -- cycle ;
%Straight Lines [id:da7163027701373094] 
\draw [color={rgb, 255:red, 0; green, 13; blue, 255 } ,draw opacity=1
] (250.37,181.61) -- (233.46,195.22) ; \draw [shift={(231.9,196.48)},
  rotate = 321.15999999999997] [color={rgb, 255:red, 0; green, 13;
    blue, 255 } ,draw opacity=1 ][line width=0.75] (6.56,-1.97)
.. controls (4.17,-0.84) and (1.99,-0.18) .. (0,0) .. controls
(1.99,0.18) and (4.17,0.84) .. (6.56,1.97) ;
%Straight Lines [id:da27099612271830587] 
\draw [color={rgb, 255:red, 65; green, 117; blue, 5 } ,draw opacity=1
] (256.82,188.49) -- (247.21,198.4) ; \draw [shift={(245.82,199.84)},
  rotate = 314.11] [color={rgb, 255:red, 65; green, 117; blue, 5 }
  ,draw opacity=1 ][line width=0.75] (6.56,-1.97) .. controls
(4.17,-0.84) and (1.99,-0.18) .. (0,0) .. controls (1.99,0.18) and
(4.17,0.84) .. (6.56,1.97) ;
%Straight Lines [id:da882164603438807] 
\draw [color={rgb, 255:red, 4; green, 0; blue, 255 } ,draw opacity=1 ]
(290.5,185.75) -- (306.35,200.58) ; \draw [shift={(307.81,201.95)},
  rotate = 223.09] [color={rgb, 255:red, 4; green, 0; blue, 255 }
  ,draw opacity=1 ][line width=0.75] (6.56,-1.97) .. controls
(4.17,-0.84) and (1.99,-0.18) .. (0,0) .. controls (1.99,0.18) and
(4.17,0.84) .. (6.56,1.97) ;
%Straight Lines [id:da884690742053359] 
\draw [color={rgb, 255:red, 65; green, 117; blue, 5 } ,draw opacity=1
] (296.41,178.39) -- (307.57,186.52) ; \draw [shift={(309.19,187.7)},
  rotate = 216.04] [color={rgb, 255:red, 65; green, 117; blue, 5 }
  ,draw opacity=1 ][line width=0.75] (6.56,-1.97) .. controls
(4.17,-0.84) and (1.99,-0.18) .. (0,0) .. controls (1.99,0.18) and
(4.17,0.84) .. (6.56,1.97) ;
%Straight Lines [id:da20074876184087243] 
\draw [color={rgb, 255:red, 41; green, 51; blue, 241 } ,draw opacity=1
] (270.3,191.49) -- (270.4,213.19) ; \draw [shift={(270.41,215.19)},
  rotate = 269.72] [color={rgb, 255:red, 41; green, 51; blue, 241 }
  ,draw opacity=1 ][line width=0.75] (6.56,-1.97) .. controls
(4.17,-0.84) and (1.99,-0.18) .. (0,0) .. controls (1.99,0.18) and
(4.17,0.84) .. (6.56,1.97) ;
%Straight Lines [id:da608342686522136] 
\draw [color={rgb, 255:red, 65; green, 117; blue, 5 } ,draw opacity=1
] (279.7,190.73) -- (281.46,204.42) ; \draw [shift={(281.72,206.41)},
  rotate = 262.67] [color={rgb, 255:red, 65; green, 117; blue, 5 }
  ,draw opacity=1 ][line width=0.75] (6.56,-1.97) .. controls
(4.17,-0.84) and (1.99,-0.18) .. (0,0) .. controls (1.99,0.18) and
(4.17,0.84) .. (6.56,1.97) ;
%Straight Lines [id:da088624174223913] 
\draw [color={rgb, 255:red, 4; green, 0; blue, 255 } ,draw opacity=1 ]
(247.34,162.12) -- (225.65,161.27) ; \draw [shift={(223.65,161.19)},
  rotate = 362.25] [color={rgb, 255:red, 4; green, 0; blue, 255 }
  ,draw opacity=1 ][line width=0.75] (6.56,-1.97) .. controls
(4.17,-0.84) and (1.99,-0.18) .. (0,0) .. controls (1.99,0.18) and
(4.17,0.84) .. (6.56,1.97) ;
%Straight Lines [id:da3058655897728151] 
\draw [color={rgb, 255:red, 65; green, 117; blue, 5 } ,draw opacity=1
] (247.69,171.55) -- (233.92,172.7) ; \draw [shift={(231.93,172.87)},
  rotate = 355.2] [color={rgb, 255:red, 65; green, 117; blue, 5 }
  ,draw opacity=1 ][line width=0.75] (6.56,-1.97) .. controls
(4.17,-0.84) and (1.99,-0.18) .. (0,0) .. controls (1.99,0.18) and
(4.17,0.84) .. (6.56,1.97) ;
%Straight Lines [id:da7992511633354091] 
\draw [color={rgb, 255:red, 4; green, 0; blue, 255 } ,draw opacity=1 ]
(296.59,151.57) -- (313.23,137.63) ; \draw [shift={(314.76,136.34)},
  rotate = 500.02] [color={rgb, 255:red, 4; green, 0; blue, 255 }
  ,draw opacity=1 ][line width=0.75] (6.56,-1.97) .. controls
(4.17,-0.84) and (1.99,-0.18) .. (0,0) .. controls (1.99,0.18) and
(4.17,0.84) .. (6.56,1.97) ;
%Straight Lines [id:da021188728886464392] 
\draw [color={rgb, 255:red, 65; green, 117; blue, 5 } ,draw opacity=1
] (290,144.82) -- (299.42,134.72) ; \draw [shift={(300.78,133.26)},
  rotate = 492.97] [color={rgb, 255:red, 65; green, 117; blue, 5 }
  ,draw opacity=1 ][line width=0.75] (6.56,-1.97) .. controls
(4.17,-0.84) and (1.99,-0.18) .. (0,0) .. controls (1.99,0.18) and
(4.17,0.84) .. (6.56,1.97) ;
%Straight Lines [id:da5888482352950647] 
\draw [color={rgb, 255:red, 4; green, 0; blue, 255 } ,draw opacity=1 ]
(254.98,145.05) -- (240.02,129.33) ; \draw [shift={(238.64,127.88)},
  rotate = 406.4] [color={rgb, 255:red, 4; green, 0; blue, 255 } ,draw
  opacity=1 ][line width=0.75] (6.56,-1.97) .. controls (4.17,-0.84)
and (1.99,-0.18) .. (0,0) .. controls (1.99,0.18) and (4.17,0.84)
.. (6.56,1.97) ;
%Straight Lines [id:da4986041081219512] 
\draw [color={rgb, 255:red, 65; green, 117; blue, 5 } ,draw opacity=1
] (248.67,152.06) -- (237.99,143.3) ; \draw [shift={(236.44,142.03)},
  rotate = 399.35] [color={rgb, 255:red, 65; green, 117; blue, 5 }
  ,draw opacity=1 ][line width=0.75] (6.56,-1.97) .. controls
(4.17,-0.84) and (1.99,-0.18) .. (0,0) .. controls (1.99,0.18) and
(4.17,0.84) .. (6.56,1.97) ;
%Straight Lines [id:da8801881465271447] 
\draw [color={rgb, 255:red, 0; green, 13; blue, 255 } ,draw opacity=1
] (276.23,135.96) -- (277.24,114.27) ; \draw [shift={(277.33,112.28)},
  rotate = 452.65] [color={rgb, 255:red, 0; green, 13; blue, 255 }
  ,draw opacity=1 ][line width=0.75] (6.56,-1.97) .. controls
(4.17,-0.84) and (1.99,-0.18) .. (0,0) .. controls (1.99,0.18) and
(4.17,0.84) .. (6.56,1.97) ;
%Straight Lines [id:da03421807598256077] 
\draw [color={rgb, 255:red, 65; green, 117; blue, 5 } ,draw opacity=1
] (266.8,136.24) -- (265.75,122.47) ; \draw [shift={(265.59,120.47)},
  rotate = 445.6] [color={rgb, 255:red, 65; green, 117; blue, 5 }
  ,draw opacity=1 ][line width=0.75] (6.56,-1.97) .. controls
(4.17,-0.84) and (1.99,-0.18) .. (0,0) .. controls (1.99,0.18) and
(4.17,0.84) .. (6.56,1.97) ;
%Straight Lines [id:da42029107361898876] 
\draw [color={rgb, 255:red, 4; green, 0; blue, 255 } ,draw opacity=1 ]
(297.28,170.57) -- (318.97,171.32) ; \draw [shift={(320.97,171.39)},
  rotate = 181.97] [color={rgb, 255:red, 4; green, 0; blue, 255 }
  ,draw opacity=1 ][line width=0.75] (6.56,-1.97) .. controls
(4.17,-0.84) and (1.99,-0.18) .. (0,0) .. controls (1.99,0.18) and
(4.17,0.84) .. (6.56,1.97) ;
%Straight Lines [id:da34990856882657484] 
\draw [color={rgb, 255:red, 65; green, 117; blue, 5 } ,draw opacity=1
] (296.89,161.15) -- (310.64,159.92) ; \draw [shift={(312.64,159.75)},
  rotate = 534.9200000000001] [color={rgb, 255:red, 65; green, 117;
    blue, 5 } ,draw opacity=1 ][line width=0.75] (6.56,-1.97)
.. controls (4.17,-0.84) and (1.99,-0.18) .. (0,0) .. controls
(1.99,0.18) and (4.17,0.84) .. (6.56,1.97) ;
%Straight Lines [id:da8568463836228528] 
\draw [color={rgb, 255:red, 0; green, 13; blue, 255 } ,draw opacity=1
] (219.17,206) -- (205.78,215.82) ; \draw [shift={(204.17,217)},
  rotate = 323.75] [color={rgb, 255:red, 0; green, 13; blue, 255 }
  ,draw opacity=1 ][line width=0.75] (6.56,-1.97) .. controls
(4.17,-0.84) and (1.99,-0.18) .. (0,0) .. controls (1.99,0.18) and
(4.17,0.84) .. (6.56,1.97) ;
%Straight Lines [id:da6974865872969185] 
\draw [color={rgb, 255:red, 65; green, 117; blue, 5 } ,draw opacity=1
] (233.08,216.87) -- (226.48,224.49) ; \draw [shift={(225.17,226)},
  rotate = 310.94] [color={rgb, 255:red, 65; green, 117; blue, 5 }
  ,draw opacity=1 ][line width=0.75] (6.56,-1.97) .. controls
(4.17,-0.84) and (1.99,-0.18) .. (0,0) .. controls (1.99,0.18) and
(4.17,0.84) .. (6.56,1.97) ;
%Straight Lines [id:da9473192312247024] 
\draw [color={rgb, 255:red, 4; green, 0; blue, 255 } ,draw opacity=1 ]
(319.86,212.07) -- (329.76,222.08) ; \draw [shift={(331.17,223.5)},
  rotate = 225.31] [color={rgb, 255:red, 4; green, 0; blue, 255 }
  ,draw opacity=1 ][line width=0.75] (6.56,-1.97) .. controls
(4.17,-0.84) and (1.99,-0.18) .. (0,0) .. controls (1.99,0.18) and
(4.17,0.84) .. (6.56,1.97) ;
%Straight Lines [id:da5546786921432448] 
\draw [color={rgb, 255:red, 65; green, 117; blue, 5 } ,draw opacity=1
] (330.42,198.47) -- (338.39,202.58) ; \draw [shift={(340.17,203.5)},
  rotate = 207.31] [color={rgb, 255:red, 65; green, 117; blue, 5 }
  ,draw opacity=1 ][line width=0.75] (6.56,-1.97) .. controls
(4.17,-0.84) and (1.99,-0.18) .. (0,0) .. controls (1.99,0.18) and
(4.17,0.84) .. (6.56,1.97) ;
%Straight Lines [id:da2980392776012871] 
\draw [color={rgb, 255:red, 4; green, 0; blue, 255 } ,draw opacity=1 ]
(208.09,160.96) -- (192.16,160.11) ; \draw [shift={(190.17,160)},
  rotate = 363.07] [color={rgb, 255:red, 4; green, 0; blue, 255 }
  ,draw opacity=1 ][line width=0.75] (6.56,-1.97) .. controls
(4.17,-0.84) and (1.99,-0.18) .. (0,0) .. controls (1.99,0.18) and
(4.17,0.84) .. (6.56,1.97) ;
%Straight Lines [id:da3672184968558653] 
\draw [color={rgb, 255:red, 65; green, 117; blue, 5 } ,draw opacity=1
] (208,173.95) -- (198.17,173.99) ; \draw [shift={(196.17,174)},
  rotate = 359.77] [color={rgb, 255:red, 65; green, 117; blue, 5 }
  ,draw opacity=1 ][line width=0.75] (6.56,-1.97) .. controls
(4.17,-0.84) and (1.99,-0.18) .. (0,0) .. controls (1.99,0.18) and
(4.17,0.84) .. (6.56,1.97) ;
%Straight Lines [id:da11017355529948814] 
\draw [color={rgb, 255:red, 4; green, 0; blue, 255 } ,draw opacity=1 ]
(325.09,129.6) -- (335.44,123.51) ; \draw [shift={(337.17,122.5)},
  rotate = 509.55] [color={rgb, 255:red, 4; green, 0; blue, 255 }
  ,draw opacity=1 ][line width=0.75] (6.56,-1.97) .. controls
(4.17,-0.84) and (1.99,-0.18) .. (0,0) .. controls (1.99,0.18) and
(4.17,0.84) .. (6.56,1.97) ;
%Straight Lines [id:da6955212381544827] 
\draw [color={rgb, 255:red, 65; green, 117; blue, 5 } ,draw opacity=1
] (312.53,116.67) -- (319.71,109.87) ; \draw [shift={(321.17,108.5)},
  rotate = 496.58] [color={rgb, 255:red, 65; green, 117; blue, 5 }
  ,draw opacity=1 ][line width=0.75] (6.56,-1.97) .. controls
(4.17,-0.84) and (1.99,-0.18) .. (0,0) .. controls (1.99,0.18) and
(4.17,0.84) .. (6.56,1.97) ;
%Straight Lines [id:da3603103334401869] 
\draw [color={rgb, 255:red, 4; green, 0; blue, 255 } ,draw opacity=1 ]
(232.17,118) -- (218.49,102.5) ; \draw [shift={(217.17,101)}, rotate =
  408.58000000000004] [color={rgb, 255:red, 4; green, 0; blue, 255 }
  ,draw opacity=1 ][line width=0.75] (6.56,-1.97) .. controls
(4.17,-0.84) and (1.99,-0.18) .. (0,0) .. controls (1.99,0.18) and
(4.17,0.84) .. (6.56,1.97) ;
%Straight Lines [id:da6369305975573354] 
\draw [color={rgb, 255:red, 65; green, 117; blue, 5 } ,draw opacity=1
] (220.59,128.62) -- (212.8,123.15) ; \draw [shift={(211.17,122)},
  rotate = 395.12] [color={rgb, 255:red, 65; green, 117; blue, 5 }
  ,draw opacity=1 ][line width=0.75] (6.56,-1.97) .. controls
(4.17,-0.84) and (1.99,-0.18) .. (0,0) .. controls (1.99,0.18) and
(4.17,0.84) .. (6.56,1.97) ;
%Straight Lines [id:da3610436111733121] 
\draw [color={rgb, 255:red, 0; green, 13; blue, 255 } ,draw opacity=1
] (278.17,102) -- (279.07,84) ; \draw [shift={(279.17,82)}, rotate =
  452.86] [color={rgb, 255:red, 0; green, 13; blue, 255 } ,draw
  opacity=1 ][line width=0.75] (6.56,-1.97) .. controls (4.17,-0.84)
and (1.99,-0.18) .. (0,0) .. controls (1.99,0.18) and (4.17,0.84)
.. (6.56,1.97) ;
%Straight Lines [id:da8453237207742365] 
\draw [color={rgb, 255:red, 65; green, 117; blue, 5 } ,draw opacity=1
] (263.17,103) -- (262.39,95.99) ; \draw [shift={(262.17,94)}, rotate
  = 443.66] [color={rgb, 255:red, 65; green, 117; blue, 5 } ,draw
  opacity=1 ][line width=0.75] (6.56,-1.97) .. controls (4.17,-0.84)
and (1.99,-0.18) .. (0,0) .. controls (1.99,0.18) and (4.17,0.84)
.. (6.56,1.97) ;
%Straight Lines [id:da8500846118815689] 
\draw [color={rgb, 255:red, 4; green, 0; blue, 255 } ,draw opacity=1 ]
(335.91,171.88) -- (347.17,172.41) ; \draw [shift={(349.17,172.5)},
  rotate = 182.66] [color={rgb, 255:red, 4; green, 0; blue, 255 }
  ,draw opacity=1 ][line width=0.75] (6.56,-1.97) .. controls
(4.17,-0.84) and (1.99,-0.18) .. (0,0) .. controls (1.99,0.18) and
(4.17,0.84) .. (6.56,1.97) ;
%Straight Lines [id:da35490233668558013] 
\draw [color={rgb, 255:red, 65; green, 117; blue, 5 } ,draw opacity=1
] (337.69,158.9) -- (345.19,157.79) ; \draw [shift={(347.17,157.5)},
  rotate = 531.61] [color={rgb, 255:red, 65; green, 117; blue, 5 }
  ,draw opacity=1 ][line width=0.75] (6.56,-1.97) .. controls
(4.17,-0.84) and (1.99,-0.18) .. (0,0) .. controls (1.99,0.18) and
(4.17,0.84) .. (6.56,1.97) ;
%Straight Lines [id:da09061677521708489] 
\draw [color={rgb, 255:red, 4; green, 0; blue, 255 } ,draw opacity=1 ]
(270.17,232) -- (269.63,246.1) ; \draw [shift={(269.55,248.1)}, rotate
  = 272.19] [color={rgb, 255:red, 4; green, 0; blue, 255 } ,draw
  opacity=1 ][line width=0.75] (6.56,-1.97) .. controls (4.17,-0.84)
and (1.99,-0.18) .. (0,0) .. controls (1.99,0.18) and (4.17,0.84)
.. (6.56,1.97) ;
%Straight Lines [id:da33913903399046463] 
\draw [color={rgb, 255:red, 65; green, 117; blue, 5 } ,draw opacity=1
] (285.64,228.91) -- (287.72,238.05) ; \draw [shift={(288.17,240)},
  rotate = 257.17] [color={rgb, 255:red, 65; green, 117; blue, 5 }
  ,draw opacity=1 ][line width=0.75] (6.56,-1.97) .. controls
(4.17,-0.84) and (1.99,-0.18) .. (0,0) .. controls (1.99,0.18) and
(4.17,0.84) .. (6.56,1.97) ;


% Text Node
\draw (204,138.4) node [anchor=north west][inner sep=0.75pt] {$\Sigma
  $};
% Text Node
\draw (273.23,157.14) node  [font=\scriptsize]  {$\boldsymbol{Q}$};


\end{tikzpicture}

\caption{Representación de la acumulación de carga en una región
  $\mathcal{R}$ especifica dentro de un conductor homogéneo, rodeada
  por una superficie $\Sigma $ ficticia sobre la cual se aplica la ley
  de Gauss al campo producido por $Q$, campo que produce una densidad
  de corriente que fluye a través de $\Sigma $. El flujo de carga
  modifica $Q$ en el tiempo $t$ produciendo las llamadas oscilaciones
  de plasma.}
\end{figure}

\begin{equation}
  \label{ChargeDifEq}
  \frac{d^{2}Q}{dt^{2}}=-\frac{4\pi n e^{2}}{m}Q,
\end{equation}
la cual es una ecuación diferencial idéntica a la ecuación de un
oscilador armónico simple como el que si ilustra en la
imagen \ref{OscArmonicoSimple}, cuya derivación se muestra en seguida,
\begin{multicols}{2}
\begin{figure}[H]
      
\begin{tikzpicture}[x=0.9pt,y=0.9pt,yscale=-1,xscale=1]
%uncomment if require: \path (0,193); %set diagram left start at 0,
%and has height of 193

%Shape: Inductor (Air Core) [id:dp34048340969550106] 
\draw (119.21,42.46) -- (119.25,59.21) .. controls (131.85,59.42) and
(142.62,61.72) .. (146.39,65.01) .. controls (150.17,68.29) and
(146.17,71.88) .. (136.32,74.06) .. controls (128.64,75.74) and
(118.71,76.44) .. (109.06,75.98) .. controls (105.3,75.99) and
(102.24,75.16) .. (102.24,74.14) .. controls (102.24,73.11) and
(105.29,72.27) .. (109.05,72.26) .. controls (118.7,71.76) and
(128.63,72.42) .. (136.32,74.06) .. controls (144.52,75.98) and
(149.16,78.66) .. (149.17,81.48) .. controls (149.18,84.29) and
(144.54,87) .. (136.36,88.95) .. controls (128.68,90.63) and
(118.74,91.33) .. (109.09,90.87) .. controls (105.33,90.88) and
(102.27,90.05) .. (102.27,89.03) .. controls (102.27,88) and
(105.32,87.16) .. (109.08,87.15) .. controls (118.73,86.65) and
(128.67,87.3) .. (136.36,88.95) .. controls (144.55,90.86) and
(149.2,93.55) .. (149.2,96.36) .. controls (149.21,99.18) and
(144.57,101.88) .. (136.39,103.84) .. controls (128.71,105.52) and
(118.77,106.22) .. (109.13,105.76) .. controls (105.36,105.77) and
(102.31,104.94) .. (102.31,103.91) .. controls (102.3,102.89) and
(105.36,102.05) .. (109.12,102.04) .. controls (118.76,101.54) and
(128.7,102.19) .. (136.39,103.84) .. controls (146.25,105.97) and
(150.26,109.55) .. (146.5,112.85) .. controls (142.74,116.15) and
(131.98,118.49) .. (119.38,118.76) -- (119.42,135.51) ;
%Shape: Rectangle [id:dp47577099884182095] 
\draw [fill={rgb, 255:red, 245; green, 166; blue, 35 } ,fill opacity=1
] (87,27.5) -- (151.17,27.5) -- (151.17,43) -- (87,43) -- cycle ;
%Shape: Circle [id:dp24543618839132486] 
\draw [fill={rgb, 255:red, 126; green, 211; blue, 33 } ,fill opacity=1
] (109.67,145.26) .. controls (109.67,139.88) and (114.04,135.51)
.. (119.42,135.51) .. controls (124.8,135.51) and (129.17,139.88)
.. (129.17,145.26) .. controls (129.17,150.65) and (124.8,155.01)
.. (119.42,155.01) .. controls (114.04,155.01) and (109.67,150.65)
.. (109.67,145.26) -- cycle ;
%Shape: Inductor (Air Core) [id:dp8275323218505621] 
\draw (42.21,42.46) -- (42.23,51.56) .. controls (54.83,51.66) and
(65.6,52.9) .. (69.37,54.68) .. controls (73.14,56.46) and
(69.14,58.41) .. (59.29,59.61) .. controls (51.61,60.53) and
(41.67,60.92) .. (32.02,60.68) .. controls (28.26,60.69) and
(25.21,60.24) .. (25.21,59.68) .. controls (25.2,59.13) and
(28.26,58.67) .. (32.02,58.66) .. controls (41.67,58.37) and
(51.6,58.72) .. (59.29,59.61) .. controls (67.48,60.64) and
(72.12,62.09) .. (72.13,63.62) .. controls (72.13,65.15) and
(67.49,66.62) .. (59.31,67.69) .. controls (51.63,68.61) and
(41.69,69) .. (32.04,68.76) .. controls (28.28,68.77) and
(25.23,68.33) .. (25.22,67.77) .. controls (25.22,67.21) and
(28.28,66.75) .. (32.04,66.74) .. controls (41.68,66.46) and
(51.62,66.81) .. (59.31,67.69) .. controls (67.5,68.72) and
(72.14,70.18) .. (72.15,71.71) .. controls (72.15,73.24) and
(67.51,74.71) .. (59.33,75.78) .. controls (51.64,76.7) and
(41.71,77.09) .. (32.06,76.85) .. controls (28.3,76.86) and
(25.24,76.41) .. (25.24,75.86) .. controls (25.24,75.3) and
(28.29,74.84) .. (32.06,74.83) .. controls (41.7,74.55) and
(51.64,74.89) .. (59.33,75.78) .. controls (69.18,76.93) and
(73.19,78.86) .. (69.43,80.66) .. controls (65.67,82.46) and
(54.9,83.74) .. (42.3,83.9) -- (42.32,93) ;
%Shape: Rectangle [id:dp7392561467625306] 
\draw [fill={rgb, 255:red, 245; green, 166; blue, 35 } ,fill opacity=1
] (10,27.5) -- (74.17,27.5) -- (74.17,43) -- (10,43) -- cycle ;
%Shape: Circle [id:dp15959587945734588] 
\draw [fill={rgb, 255:red, 126; green, 211; blue, 33 } ,fill opacity=1
] (32.57,102.75) .. controls (32.57,97.36) and (36.94,93)
.. (42.32,93) .. controls (47.71,93) and (52.07,97.36)
.. (52.07,102.75) .. controls (52.07,108.13) and (47.71,112.5)
.. (42.32,112.5) .. controls (36.94,112.5) and (32.57,108.13)
.. (32.57,102.75) -- cycle ;

%Straight Lines [id:da8898992399758622] 
\draw (95.32,102.75) -- (95.17,147) ; \draw [shift={(95.17,147)},
  rotate = 270.2] [color={rgb, 255:red, 0; green, 0; blue, 0 } ][line
  width=0.75] (0,5.59) -- (0,-5.59) ; \draw [shift={(95.32,102.75)},
  rotate = 270.2] [color={rgb, 255:red, 0; green, 0; blue, 0 } ][line
  width=0.75] (0,5.59) -- (0,-5.59) ;


% Text Node
\draw (5,58.4) node [anchor=north west][inner sep=0.75pt]    {$k$};
% Text Node
\draw (57,100.4) node [anchor=north west][inner sep=0.75pt]    {$m$};
% Text Node
\draw (132,134.4) node [anchor=north west][inner sep=0.75pt]    {$m$};
% Text Node
\draw (83,113.4) node [anchor=north west][inner sep=0.75pt]    {$y$};


\end{tikzpicture}
\label{OscArmonicoSimple}
\caption{Oscilador armónico en equilibrio y acelerado.}
\end{figure}

\begin{equation}
  \begin{split}
    \vec{F} &= -k\vec{r} \\
    m\vec{a}&=\vec{F} \\
    m\frac{d^{2}y}{dt^{2}}&=-ky,\\
    y(t)&=y_{0}\cos(\omega t),\\
    \omega^{2} &= k/m,\\
    \frac{d^{2}y}{dt^{2}}&=-\omega^{2} y,
    \label{OAEq}
  \end{split}
\end{equation} 

\end{multicols}

con lo que se obtiene la frecuencia de plasma como,
\begin{equation}
  \label{plasmafec}
  \omega _{p}^{2} = \frac{4\pi ne^{2}}{m}.
\end{equation}

Con un análisis análogo podemos obtener la frecuencia de un plasmón de
superficie, considerando al cumulo de carga $Q$ en una región $R$ en
la interfaz de un conductor semiinfinito como ilustra la imagen
\ref{Surfplasmon},
\begin{figure}[H]
  \centering
\tikzset{every picture/.style={line width=0.75pt}} %set default line width to 0.75pt        

\begin{tikzpicture}[x=0.75pt,y=0.75pt,yscale=-1,xscale=1]
%uncomment if require: \path (0,300); %set diagram left start at 0,
%and has height of 300

%Shape: Rectangle [id:dp7956545958448051] 
\draw [draw opacity=0][fill={rgb, 255:red, 198; green, 244; blue, 240
  } ,fill opacity=0.48 ] (114.02,144) -- (356.19,144) --
(356.19,248.02) -- (114.02,248.02) -- cycle ;
%Shape: Circle [id:dp44476661130132666] 
\draw [dash pattern={on 0.84pt off 2.51pt}] (180.04,142.52)
.. controls (180.04,111.56) and (205.14,86.46) .. (236.1,86.46)
.. controls (267.07,86.46) and (292.17,111.56) .. (292.17,142.52)
.. controls (292.17,173.48) and (267.07,198.58) .. (236.1,198.58)
.. controls (205.14,198.58) and (180.04,173.48) .. (180.04,142.52) --
cycle ;
%Straight Lines [id:da14033928295269626] 
\draw [color={rgb, 255:red, 0; green, 13; blue, 255 } ,draw opacity=1
] (213.37,157.61) -- (196.46,171.22) ; \draw [shift={(194.9,172.48)},
  rotate = 321.15999999999997] [color={rgb, 255:red, 0; green, 13;
    blue, 255 } ,draw opacity=1 ][line width=0.75] (6.56,-1.97)
.. controls (4.17,-0.84) and (1.99,-0.18) .. (0,0) .. controls
(1.99,0.18) and (4.17,0.84) .. (6.56,1.97) ;
%Straight Lines [id:da45124177831622647] 
\draw [color={rgb, 255:red, 65; green, 117; blue, 5 } ,draw opacity=1
] (219.82,164.49) -- (210.21,174.4) ; \draw [shift={(208.82,175.84)},
  rotate = 314.11] [color={rgb, 255:red, 65; green, 117; blue, 5 }
  ,draw opacity=1 ][line width=0.75] (6.56,-1.97) .. controls
(4.17,-0.84) and (1.99,-0.18) .. (0,0) .. controls (1.99,0.18) and
(4.17,0.84) .. (6.56,1.97) ;
%Straight Lines [id:da044509622606934585] 
\draw [color={rgb, 255:red, 4; green, 0; blue, 255 } ,draw opacity=1 ]
(253.5,161.75) -- (269.35,176.58) ; \draw [shift={(270.81,177.95)},
  rotate = 223.09] [color={rgb, 255:red, 4; green, 0; blue, 255 }
  ,draw opacity=1 ][line width=0.75] (6.56,-1.97) .. controls
(4.17,-0.84) and (1.99,-0.18) .. (0,0) .. controls (1.99,0.18) and
(4.17,0.84) .. (6.56,1.97) ;
%Straight Lines [id:da16284278441076816] 
\draw [color={rgb, 255:red, 65; green, 117; blue, 5 } ,draw opacity=1
] (259.41,154.39) -- (270.57,162.52) ; \draw [shift={(272.19,163.7)},
  rotate = 216.04] [color={rgb, 255:red, 65; green, 117; blue, 5 }
  ,draw opacity=1 ][line width=0.75] (6.56,-1.97) .. controls
(4.17,-0.84) and (1.99,-0.18) .. (0,0) .. controls (1.99,0.18) and
(4.17,0.84) .. (6.56,1.97) ;
%Straight Lines [id:da8217139032715733] 
\draw [color={rgb, 255:red, 41; green, 51; blue, 241 } ,draw opacity=1
] (233.3,167.49) -- (233.4,189.19) ; \draw [shift={(233.41,191.19)},
  rotate = 269.72] [color={rgb, 255:red, 41; green, 51; blue, 241 }
  ,draw opacity=1 ][line width=0.75] (6.56,-1.97) .. controls
(4.17,-0.84) and (1.99,-0.18) .. (0,0) .. controls (1.99,0.18) and
(4.17,0.84) .. (6.56,1.97) ;
%Straight Lines [id:da4291710065118748] 
\draw [color={rgb, 255:red, 65; green, 117; blue, 5 } ,draw opacity=1
] (242.7,166.73) -- (244.46,180.42) ; \draw [shift={(244.72,182.41)},
  rotate = 262.67] [color={rgb, 255:red, 65; green, 117; blue, 5 }
  ,draw opacity=1 ][line width=0.75] (6.56,-1.97) .. controls
(4.17,-0.84) and (1.99,-0.18) .. (0,0) .. controls (1.99,0.18) and
(4.17,0.84) .. (6.56,1.97) ;
%Straight Lines [id:da8972267016384223] 
\draw [color={rgb, 255:red, 4; green, 0; blue, 255 } ,draw opacity=1 ]
(210.34,138.12) -- (188.65,137.27) ; \draw [shift={(186.65,137.19)},
  rotate = 362.25] [color={rgb, 255:red, 4; green, 0; blue, 255 }
  ,draw opacity=1 ][line width=0.75] (6.56,-1.97) .. controls
(4.17,-0.84) and (1.99,-0.18) .. (0,0) .. controls (1.99,0.18) and
(4.17,0.84) .. (6.56,1.97) ;
%Straight Lines [id:da813137464274293] 
\draw [color={rgb, 255:red, 65; green, 117; blue, 5 } ,draw opacity=1
] (210.69,147.55) -- (196.92,148.7) ; \draw [shift={(194.93,148.87)},
  rotate = 355.2] [color={rgb, 255:red, 65; green, 117; blue, 5 }
  ,draw opacity=1 ][line width=0.75] (6.56,-1.97) .. controls
(4.17,-0.84) and (1.99,-0.18) .. (0,0) .. controls (1.99,0.18) and
(4.17,0.84) .. (6.56,1.97) ;
%Straight Lines [id:da3182154012798515] 
\draw [color={rgb, 255:red, 4; green, 0; blue, 255 } ,draw opacity=1 ]
(259.59,127.57) -- (276.23,113.63) ; \draw [shift={(277.76,112.34)},
  rotate = 500.02] [color={rgb, 255:red, 4; green, 0; blue, 255 }
  ,draw opacity=1 ][line width=0.75] (6.56,-1.97) .. controls
(4.17,-0.84) and (1.99,-0.18) .. (0,0) .. controls (1.99,0.18) and
(4.17,0.84) .. (6.56,1.97) ;
%Straight Lines [id:da18228076317796438] 
\draw [color={rgb, 255:red, 65; green, 117; blue, 5 } ,draw opacity=1
] (253,120.82) -- (262.42,110.72) ; \draw [shift={(263.78,109.26)},
  rotate = 492.97] [color={rgb, 255:red, 65; green, 117; blue, 5 }
  ,draw opacity=1 ][line width=0.75] (6.56,-1.97) .. controls
(4.17,-0.84) and (1.99,-0.18) .. (0,0) .. controls (1.99,0.18) and
(4.17,0.84) .. (6.56,1.97) ;
%Straight Lines [id:da6141206217382678] 
\draw [color={rgb, 255:red, 4; green, 0; blue, 255 } ,draw opacity=1 ]
(217.98,121.05) -- (203.02,105.33) ; \draw [shift={(201.64,103.88)},
  rotate = 406.4] [color={rgb, 255:red, 4; green, 0; blue, 255 } ,draw
  opacity=1 ][line width=0.75] (6.56,-1.97) .. controls (4.17,-0.84)
and (1.99,-0.18) .. (0,0) .. controls (1.99,0.18) and (4.17,0.84)
.. (6.56,1.97) ;
%Straight Lines [id:da8166253971322168] 
\draw [color={rgb, 255:red, 65; green, 117; blue, 5 } ,draw opacity=1
] (211.67,128.06) -- (200.99,119.3) ; \draw [shift={(199.44,118.03)},
  rotate = 399.35] [color={rgb, 255:red, 65; green, 117; blue, 5 }
  ,draw opacity=1 ][line width=0.75] (6.56,-1.97) .. controls
(4.17,-0.84) and (1.99,-0.18) .. (0,0) .. controls (1.99,0.18) and
(4.17,0.84) .. (6.56,1.97) ;
%Straight Lines [id:da3465118382665756] 
\draw [color={rgb, 255:red, 0; green, 13; blue, 255 } ,draw opacity=1
] (239.23,111.96) -- (240.24,90.27) ; \draw [shift={(240.33,88.28)},
  rotate = 452.65] [color={rgb, 255:red, 0; green, 13; blue, 255 }
  ,draw opacity=1 ][line width=0.75] (6.56,-1.97) .. controls
(4.17,-0.84) and (1.99,-0.18) .. (0,0) .. controls (1.99,0.18) and
(4.17,0.84) .. (6.56,1.97) ;
%Straight Lines [id:da9358764629291445] 
\draw [color={rgb, 255:red, 65; green, 117; blue, 5 } ,draw opacity=1
] (229.8,112.24) -- (228.75,98.47) ; \draw [shift={(228.59,96.47)},
  rotate = 445.6] [color={rgb, 255:red, 65; green, 117; blue, 5 }
  ,draw opacity=1 ][line width=0.75] (6.56,-1.97) .. controls
(4.17,-0.84) and (1.99,-0.18) .. (0,0) .. controls (1.99,0.18) and
(4.17,0.84) .. (6.56,1.97) ;
%Straight Lines [id:da5941951041547017] 
\draw [color={rgb, 255:red, 4; green, 0; blue, 255 } ,draw opacity=1 ]
(260.28,146.57) -- (281.97,147.32) ; \draw [shift={(283.97,147.39)},
  rotate = 181.97] [color={rgb, 255:red, 4; green, 0; blue, 255 }
  ,draw opacity=1 ][line width=0.75] (6.56,-1.97) .. controls
(4.17,-0.84) and (1.99,-0.18) .. (0,0) .. controls (1.99,0.18) and
(4.17,0.84) .. (6.56,1.97) ;
%Straight Lines [id:da9395957665484183] 
\draw [color={rgb, 255:red, 65; green, 117; blue, 5 } ,draw opacity=1
] (259.89,137.15) -- (273.64,135.92) ; \draw [shift={(275.64,135.75)},
  rotate = 534.9200000000001] [color={rgb, 255:red, 65; green, 117;
    blue, 5 } ,draw opacity=1 ][line width=0.75] (6.56,-1.97)
.. controls (4.17,-0.84) and (1.99,-0.18) .. (0,0) .. controls
(1.99,0.18) and (4.17,0.84) .. (6.56,1.97) ;
%Straight Lines [id:da6754579892739427] 
\draw [color={rgb, 255:red, 0; green, 13; blue, 255 } ,draw opacity=1
] (182.17,182) -- (168.78,191.82) ; \draw [shift={(167.17,193)},
  rotate = 323.75] [color={rgb, 255:red, 0; green, 13; blue, 255 }
  ,draw opacity=1 ][line width=0.75] (6.56,-1.97) .. controls
(4.17,-0.84) and (1.99,-0.18) .. (0,0) .. controls (1.99,0.18) and
(4.17,0.84) .. (6.56,1.97) ;
%Straight Lines [id:da29198388310333045] 
\draw [color={rgb, 255:red, 65; green, 117; blue, 5 } ,draw opacity=1
] (196.08,192.87) -- (189.48,200.49) ; \draw [shift={(188.17,202)},
  rotate = 310.94] [color={rgb, 255:red, 65; green, 117; blue, 5 }
  ,draw opacity=1 ][line width=0.75] (6.56,-1.97) .. controls
(4.17,-0.84) and (1.99,-0.18) .. (0,0) .. controls (1.99,0.18) and
(4.17,0.84) .. (6.56,1.97) ;
%Straight Lines [id:da9217469948945557] 
\draw [color={rgb, 255:red, 4; green, 0; blue, 255 } ,draw opacity=1 ]
(282.86,188.07) -- (292.76,198.08) ; \draw [shift={(294.17,199.5)},
  rotate = 225.31] [color={rgb, 255:red, 4; green, 0; blue, 255 }
  ,draw opacity=1 ][line width=0.75] (6.56,-1.97) .. controls
(4.17,-0.84) and (1.99,-0.18) .. (0,0) .. controls (1.99,0.18) and
(4.17,0.84) .. (6.56,1.97) ;
%Straight Lines [id:da9680792977793742] 
\draw [color={rgb, 255:red, 65; green, 117; blue, 5 } ,draw opacity=1
] (293.42,174.47) -- (301.39,178.58) ; \draw [shift={(303.17,179.5)},
  rotate = 207.31] [color={rgb, 255:red, 65; green, 117; blue, 5 }
  ,draw opacity=1 ][line width=0.75] (6.56,-1.97) .. controls
(4.17,-0.84) and (1.99,-0.18) .. (0,0) .. controls (1.99,0.18) and
(4.17,0.84) .. (6.56,1.97) ;
%Straight Lines [id:da7592534731051833] 
\draw [color={rgb, 255:red, 4; green, 0; blue, 255 } ,draw opacity=1 ]
(171.09,136.96) -- (155.16,136.11) ; \draw [shift={(153.17,136)},
  rotate = 363.07] [color={rgb, 255:red, 4; green, 0; blue, 255 }
  ,draw opacity=1 ][line width=0.75] (6.56,-1.97) .. controls
(4.17,-0.84) and (1.99,-0.18) .. (0,0) .. controls (1.99,0.18) and
(4.17,0.84) .. (6.56,1.97) ;
%Straight Lines [id:da0330093962741953] 
\draw [color={rgb, 255:red, 65; green, 117; blue, 5 } ,draw opacity=1
] (171,149.95) -- (161.17,149.99) ; \draw [shift={(159.17,150)},
  rotate = 359.77] [color={rgb, 255:red, 65; green, 117; blue, 5 }
  ,draw opacity=1 ][line width=0.75] (6.56,-1.97) .. controls
(4.17,-0.84) and (1.99,-0.18) .. (0,0) .. controls (1.99,0.18) and
(4.17,0.84) .. (6.56,1.97) ;
%Straight Lines [id:da014961923755437145] 
\draw [color={rgb, 255:red, 4; green, 0; blue, 255 } ,draw opacity=1 ]
(288.09,105.6) -- (298.44,99.51) ; \draw [shift={(300.17,98.5)},
  rotate = 509.55] [color={rgb, 255:red, 4; green, 0; blue, 255 }
  ,draw opacity=1 ][line width=0.75] (6.56,-1.97) .. controls
(4.17,-0.84) and (1.99,-0.18) .. (0,0) .. controls (1.99,0.18) and
(4.17,0.84) .. (6.56,1.97) ;
%Straight Lines [id:da054090511812718955] 
\draw [color={rgb, 255:red, 65; green, 117; blue, 5 } ,draw opacity=1
] (275.53,92.67) -- (282.71,85.87) ; \draw [shift={(284.17,84.5)},
  rotate = 496.58] [color={rgb, 255:red, 65; green, 117; blue, 5 }
  ,draw opacity=1 ][line width=0.75] (6.56,-1.97) .. controls
(4.17,-0.84) and (1.99,-0.18) .. (0,0) .. controls (1.99,0.18) and
(4.17,0.84) .. (6.56,1.97) ;
%Straight Lines [id:da20709477222668538] 
\draw [color={rgb, 255:red, 4; green, 0; blue, 255 } ,draw opacity=1 ]
(195.17,94) -- (181.49,78.5) ; \draw [shift={(180.17,77)}, rotate =
  408.58000000000004] [color={rgb, 255:red, 4; green, 0; blue, 255 }
  ,draw opacity=1 ][line width=0.75] (6.56,-1.97) .. controls
(4.17,-0.84) and (1.99,-0.18) .. (0,0) .. controls (1.99,0.18) and
(4.17,0.84) .. (6.56,1.97) ;
%Straight Lines [id:da616186312563067] 
\draw [color={rgb, 255:red, 65; green, 117; blue, 5 } ,draw opacity=1
] (183.59,104.62) -- (175.8,99.15) ; \draw [shift={(174.17,98)},
  rotate = 395.12] [color={rgb, 255:red, 65; green, 117; blue, 5 }
  ,draw opacity=1 ][line width=0.75] (6.56,-1.97) .. controls
(4.17,-0.84) and (1.99,-0.18) .. (0,0) .. controls (1.99,0.18) and
(4.17,0.84) .. (6.56,1.97) ;
%Straight Lines [id:da27027429645133494] 
\draw [color={rgb, 255:red, 0; green, 13; blue, 255 } ,draw opacity=1
] (241.17,78) -- (242.07,60) ; \draw [shift={(242.17,58)}, rotate =
  452.86] [color={rgb, 255:red, 0; green, 13; blue, 255 } ,draw
  opacity=1 ][line width=0.75] (6.56,-1.97) .. controls (4.17,-0.84)
and (1.99,-0.18) .. (0,0) .. controls (1.99,0.18) and (4.17,0.84)
.. (6.56,1.97) ;
%Straight Lines [id:da2297200609458736] 
\draw [color={rgb, 255:red, 65; green, 117; blue, 5 } ,draw opacity=1
] (226.17,79) -- (225.39,71.99) ; \draw [shift={(225.17,70)}, rotate =
  443.66] [color={rgb, 255:red, 65; green, 117; blue, 5 } ,draw
  opacity=1 ][line width=0.75] (6.56,-1.97) .. controls (4.17,-0.84)
and (1.99,-0.18) .. (0,0) .. controls (1.99,0.18) and (4.17,0.84)
.. (6.56,1.97) ;
%Straight Lines [id:da33177653749083236] 
\draw [color={rgb, 255:red, 4; green, 0; blue, 255 } ,draw opacity=1 ]
(298.91,147.88) -- (310.17,148.41) ; \draw [shift={(312.17,148.5)},
  rotate = 182.66] [color={rgb, 255:red, 4; green, 0; blue, 255 }
  ,draw opacity=1 ][line width=0.75] (6.56,-1.97) .. controls
(4.17,-0.84) and (1.99,-0.18) .. (0,0) .. controls (1.99,0.18) and
(4.17,0.84) .. (6.56,1.97) ;
%Straight Lines [id:da7070837655966179] 
\draw [color={rgb, 255:red, 65; green, 117; blue, 5 } ,draw opacity=1
] (300.69,134.9) -- (308.19,133.79) ; \draw [shift={(310.17,133.5)},
  rotate = 531.61] [color={rgb, 255:red, 65; green, 117; blue, 5 }
  ,draw opacity=1 ][line width=0.75] (6.56,-1.97) .. controls
(4.17,-0.84) and (1.99,-0.18) .. (0,0) .. controls (1.99,0.18) and
(4.17,0.84) .. (6.56,1.97) ;
%Straight Lines [id:da9465224193534777] 
\draw [color={rgb, 255:red, 4; green, 0; blue, 255 } ,draw opacity=1 ]
(233.17,208) -- (232.63,222.1) ; \draw [shift={(232.55,224.1)}, rotate
  = 272.19] [color={rgb, 255:red, 4; green, 0; blue, 255 } ,draw
  opacity=1 ][line width=0.75] (6.56,-1.97) .. controls (4.17,-0.84)
and (1.99,-0.18) .. (0,0) .. controls (1.99,0.18) and (4.17,0.84)
.. (6.56,1.97) ;
%Straight Lines [id:da700311003901801] 
\draw [color={rgb, 255:red, 65; green, 117; blue, 5 } ,draw opacity=1
] (248.64,204.91) -- (250.72,214.05) ; \draw [shift={(251.17,216)},
  rotate = 257.17] [color={rgb, 255:red, 65; green, 117; blue, 5 }
  ,draw opacity=1 ][line width=0.75] (6.56,-1.97) .. controls
(4.17,-0.84) and (1.99,-0.18) .. (0,0) .. controls (1.99,0.18) and
(4.17,0.84) .. (6.56,1.97) ;
%Straight Lines [id:da29842040596141506] 
\draw [line width=1.5] (114.02,144) -- (356.19,144) ;
%Shape: Polygon Curved [id:ds08608026786206713] 
\draw [fill={rgb, 255:red, 248; green, 231; blue, 28 } ,fill opacity=1
] (226.17,144) .. controls (233.4,145.13) and (236.17,144)
.. (243.17,144) .. controls (250.17,144) and (265.52,156.16)
.. (234.17,156) .. controls (202.82,155.84) and (218.93,142.87)
.. (226.17,144) -- cycle ;


% Text Node
\draw (167,114.4) node [anchor=north west][inner sep=0.75pt] {$\Sigma
  $};
% Text Node
\draw (243.17,149) node [font=\scriptsize] {$\boldsymbol{Q}$};
% Text Node
\draw (220,143.4) node [anchor=north west][inner sep=0.75pt]
      [font=\footnotesize] {$\mathcal{R}$};

\end{tikzpicture}
\label{Surfplasmon}
\caption{Region $\mathcal{R}$ cargada en la superficie de un conductor
  semiinfinito. La carga $Q$ produce un campo eléctrico \vec{E}(\vec{r},t)}
\end{figure}
 
\section{Teoría}

Las propiedades ópticas de estos materiales están determinadas por su
composición y geometría. Propiedades como las relaciones de dispersión
de los modos electromagnéticos que se propagan a través del material
en un sistema semiinfinito, las amplitudes de reflexión y transmisión,
y relaciones de dispersión de modos electromagnéticos en la superficie
de un sistema bordeado pueden ser expresadas en términos del operador
dieléctrico macroscópico a través de las soluciones de las ecuaciones
de Maxwell en dicho material.

Las ecuaciones de Maxwell con las cuales se asume una descripción
macroscópica de los campos en un medio son
\begin{equation}
  \label{Maxwell-eqs}
  \begin{array}{cc}
    \nabla \cdot {\bf D} = 4\pi \rho_{ex}, & \nabla \cdot {\bf B} =0, \\
    \nabla \times {\bf E} = -\frac{1}{c}\frac{\partial {\bf B}}{\partial t}, &
    \nabla \times {\bf H} = \frac{4\pi}{c}{\bf
      J}_{ex} + \frac{1}{c}\frac{\partial {\bf D}}{\partial t},
  \end{array}
\end{equation}
donde ${\bf J}_{ex}$ y $\rho_{ex}$ son la densidad de corriente y
densidad de carga externas, respectivamente. Y para calcular
respuestas ópticas de un sistema se define al operador dieléctrico
$\hat{\epsilon}_{M}$ a través de la relación
\begin{equation}
  {\bf D}_{a} = \hat{\epsilon}_{M} {\bf E}_{a}
  \label{Def-Dilectric-Op}
\end{equation}
donde ${\bf D}_{a}$ y ${\bf E}_{a}$ representan los promedios macroscópicos del
campo desplazamiento y eléctrico, respectivamente.

Sin embargo es posible definir ${\bf D}$ y ${\bf H}$ para incluir
fluctuaciones microscópicas y que sigan obedeciendo las ecuaciones de
Maxwell \eqref{Maxwell-eqs}.

Para poder describir la respuesta óptica de un sistema que incluye
fluctuaciones locales de los campos inducidas por inhomogeneidades del
sistema, consideramos primero una respuesta dieléctrica,
$\hat{\epsilon}$, microscópica
\begin{equation}
  {\bf D} = \hat{\epsilon} {\bf E},
  \label{MicroscopicDielectric-Op}
\end{equation}
la cual acopla las fluctuaciones locales de los campos ${\bf D}$ y
${\bf E}$ totales y a la cual se le aplica un procedimiento de
promediación para relacionar $\hat{\epsilon}$ con
$\hat{\epsilon}_{M}$, de tal manera que las fluctuaciones
microscópicas sean incorporadas en la respuesta macroscópica. Para tal
proposito seguimos el procedimiento descrito en
\cite{ElectromagneticResponseofSystemwithSpatialFluctuations}, por
Mochán et al.

Formalmente se definen los operadores 
\begin{equation}
  \begin{array}{cc}
    \label{operadoresPromedio-Fluctuaciones}
    \hat{P}_{a}, & \hat{P}_{f} = \hat{1} - \hat{P}_{a}, 
  \end{array}
\end{equation}
los cuales extraen la componente promedio $F_{a}=\hat{P}_{a}F$ y
fluctuante $F_{f}=\hat{P}_{f}F$ de una función $F = F_{a}+F_{f}$
arbitraria. Tales operadores son idempotentes
\begin{equation}
  \begin{array}{cc}
    \label{Op-Idempotentes}
    \hat{P}^{2}_{a}=\hat{P}_{a}, & \hat{P}^{2}_{f} =\hat{P}_{f},
  \end{array}
\end{equation}
\begin{equation}
  \hat{P}_{a}\hat{P}_{f}=\hat{P}_{f}\hat{P}_{a}=0,
\end{equation}
lo cual significa que son operadores de proyección.

Consideremos un campo eléctrico consistente de una parte promedio y
una parte fluctuante
\begin{equation}
  {\bf E} = \left(\begin{split} {\bf E}_{a} \\
    {\bf E}_{f} \end{split}\right),
\end{equation}
respresentado como en la siguiente imagen reescribiendo
\eqref{MicroscopicDielectric-Op} proyectada en los subespacios $a$ y
$f$,
\begin{equation}
  \label{CamposCompletos}
  \left(\begin{split} {\bf D}_{a} \\  {\bf D}_{f} \end{split}\right) =
  \left(\begin{matrix} \hat{\epsilon}_{aa} & \hat{\epsilon}_{af} \\
    \hat{\epsilon}_{fa} & \hat{\epsilon}_{ff} \end{matrix}\right)
  \left(\begin{split} {\bf E}_{a} \\ {\bf E}_{f} \end{split}\right),
\end{equation}
donde se ha definido
\begin{equation}
  \begin{array}{cc}
    \hat{O}_{\alpha \beta}\equiv \hat{P}_{\alpha}\hat{O}\hat{P}_{\beta}, & \alpha,\beta =a,f,
  \end{array}
\end{equation}
para cualquier operador.

Para calcular la respuesta macroscópica del sistema desacoplamos
${\bf D}_{a}$ y ${\bf E}_{f}$ de la ecuación de campos completos
\eqref{CamposCompletos}. Usando las ecuaciones de Maxwell para
los campos microscópicos se encuentra una relación entre ${\bf E}_{a}$
y ${\bf E}_{f}$,
\begin{equation}
  \begin{split}
      \nabla \times \nabla \times {\bf E} = \frac{4\pi i\omega}{c^{2}}{\bf
    j}_{ex} + \frac{\omega^{2}}{c^{2}}{\bf D} \\
    \left[ \hat{\epsilon}_{ff} - \frac{c^{2}}{\omega^{2}}(\nabla \times \nabla \times)_{ff}\right] {\bf E}_{f} = \frac{4\pi}{i \omega }{\bf j}_{f}^{ex}-\hat{\epsilon}_{fa}{\bf E}_{a}     
  \end{split}
\end{equation}
asumiendo que ${\bf J}^{ex}_{f} = 0 $, sustituimos ${\bf E}_{f}$ en la
ecuación \eqref{CamposCompletos},
\begin{equation}
  {\bf D}_{a}=\left(\hat{\epsilon}_{aa}-(\hat{\epsilon}_{af}(\hat{\epsilon}_{ff}-\frac{c^{2}}{\omega^{2}}(\nabla \times \nabla)_{ff})^{-1}\hat{\epsilon}_{fa}\right){\bf E}_{a},
  \label{MacroscopicRelation}
\end{equation}
comparando con la ecuación de campos macroscópicos obtenemos que
\begin{equation}
  \hat{\epsilon}_{M}=\hat{\epsilon}_{aa}-\hat{\epsilon}_{af}\left[(\hat{\epsilon}_{ff}-\frac{c^{2}}{\omega^{2}}(\nabla \times \nabla)_{ff})\right]^{-1}\hat{\epsilon}_{fa},
  \label{DielectricFunction}
\end{equation}
la respuesta macroscópica de un sistema está dada por el promedio de
su respuesta microscópica, más una corrección debida al acoplamiento
entre las componentes fluctuante y promedio de los campos, éste
resultado se presume exacto.

Para analizar el resultado de \eqref{DielectricFunction} nos
restringimos a una longitud de escala característica de las
fluctuaciones mucho menor a la longitud de onda. Convenitntemente se definen los  operadores de proyección longitudinal
\begin{equation}
  \hat{P}^{L} = \hat{\nabla} \hat{\nabla}^{-2} \hat{\nabla}\times,
\end{equation}
y transversal
\begin{equation}
  \hat{P}^{T}=-\hat{\nabla}\times \hat{\nabla}^{-2} \hat{\nabla}\times,
\end{equation}
tales que extraen las partes irrotacional ${\bf F}^{L}=\hat{P}^{L}{\bf
  F} $ y solenoidal ${\bf F}^{T}=\hat{P}^{T}{\bf F}$ de una campo
vectorial arbitrario. Estos operadores cumplen las siguiente
propiedades
\begin{equation}
  \begin{array}{cc}
    \hat{P}^{T}\hat{P}^{T}=\hat{P}^{T}, & \hat{P}^{L}\hat{P}^{L}=\hat{P}^{L},\\
    \hat{P}^{T}\hat{P}^{L}=\hat{P}^{L}\hat{P}^{T}=0, & \hat{P}^{T}+\hat{P}^{L}=\hat{1},
  \end{array}
\end{equation}
y communtan con $\hat{P}_{f}$ y $\hat{P}_{a}$.

Al proyectar $\hat{\mathcal{W}}^{-1}=
\left[(\hat{\epsilon}_{ff}-\frac{c^{2}}{\omega^{2}}(\nabla \times
  \nabla \times)_{ff})\right]^{-1}$ en los subespacios longitudina y
transversal
\[\hat{\mathcal{W}}^{-1}=\left[
    \begin{array}{cc}
    \mathcal{W}_{LL} &  \mathcal{W}_{LT} \\
    \mathcal{W}_{TL} &  \mathcal{W}_{TT}
    \end{array}
    \right]^{-1}, \]
y reescribiendo el termino de operadores nabla como, $(\nabla \times \nabla \times)_{ff} = \nabla ^{2}\hat{P}^{T}\hat{P}_{f}$,
tenemos
\begin{equation}
  \label{MatrizdeProyeccionesdeW}
  \left[(\hat{\epsilon}_{ff}-\frac{c^{2}}{\omega^{2}}(\nabla \times \nabla \times)_{ff})\right]^{-1} = \left[
  \begin{array}{cc}
    \hat{\epsilon}_{ff}^{LL} & \hat{\epsilon}_{ff}^{LT} \\
    \hat{\epsilon}_{ff}^{TL} &  \hat{\epsilon}_{ff}^{TT}+ \frac{c^{2}}{\omega^{2}}\nabla ^{2}\hat{P}^{T}\hat{P}_{f}
  \end{array}
  \right]^{-1}.
\end{equation}

La manera en que obtenemos las componentes de la matriz es la
siguiente, definimos una matriz y su inversa

\[ \hat{M} = \left(
\begin{array}{cc}
  \hat{A} &  \hat{B} \\
   \hat{C} &  \hat{D}
\end{array}\right),
\] 
\[ \hat{m} =\left( 
\begin{array}{cc}
   \hat{a} &  \hat{b} \\
   \hat{c} &  \hat{d}
\end{array}\right) = \hat{M}^{-1},
\] 
la cuales deben cumplir con las ecuaciones
\begin{equation}
  \begin{split}
    \hat{A} \hat{a}+ \hat{B} \hat{c} = 1, \\
    \hat{A} \hat{b}+ \hat{B} \hat{d} = 0, \\
    \hat{C} \hat{a}+ \hat{D} \hat{c} = 0, \\
    \hat{C} \hat{b}+ \hat{D} \hat{d} = 1,
  \end{split}
\end{equation}
de las cuales uno despeja
\begin{equation}
  \label{eqsalgebraicas}
  \begin{split}
    \hat{a} = \hat{A}^{-1}(1-\hat{B}\hat{D}^{-1}\hat{C}\hat{A}^{-1})^{-1}, \\
    \hat{b} = \hat{C}^{-1}(1-\hat{D}\hat{B}^{-1}\hat{A}\hat{C}^{-1})^{-1}, \\
    \hat{c} = - \hat{D}^{-1}\hat{C}\hat{A}^{-1}(1-\hat{B}\hat{D}^{-1}\hat{C}\hat{A}^{-1})^{-1}, \\
    \hat{d} = - \hat{B}^{-1}\hat{A} \hat{C}^{-1}(1-\hat{D}\hat{B}^{-1}\hat{A}\hat{C}^{-1})^{-1}.
  \end{split}
\end{equation}

Para usar este desarrollo definimos $\hat{M} = \hat{\mathcal{W}}$ por
lo que los elementos respectivos son
\[
\begin{split}
  \hat{A} = \hat{\epsilon}_{ff}^{LL}, \\
  \hat{B} = \hat{\epsilon}_{ff}^{LT}, \\
  \hat{C} = \hat{\epsilon}_{ff}^{TL}, \\
  \hat{D} = \hat{\epsilon}_{ff}^{TT}+\frac{c^{2}}{\omega^{2}}\nabla ^{2}\hat{P}^{T}\hat{P}_{f},
\end{split}
\]
y asumiendo que $\lambda^{2}/l^{2} >> ||\hat{\epsilon}||$ donde $l$ es
la longitud de escala de las fluctuaciones, y que $||(\omega^{2}/c^{2})\hat{\nabla}^{-2}\hat{P}_{f}||\approx l^{2}/\lambda ^{2}$, entonces el segundo termino del
elemento $\hat{D}$ es mucho mayor que el primero. Para hacer el cálculo más explícito lo reescribimos como
\[ \frac{c^{2}}{\omega^{2}}\nabla ^{2}\hat{P}^{T}\hat{P}_{f}(1+\frac{\omega^{2}}{c^{2}}\nabla ^{-2}\hat{P}^{T}\hat{P}_{f}\hat{\epsilon}_{ff}^{TT}) \]
y al sustituir en las ecuaciones \eqref{eqsalgebraicas}, obtenemos
\begin{equation}
  \begin{split}
    &\left[(\hat{\epsilon}_{ff}- \frac{c^{2}}{\omega^{2}}(\nabla \times
      \nabla \times)_{ff})\right]^{-1}  =
    \left[ \begin{array}{cc}
        (\hat{\epsilon}_{ff}^{LL})^{-1} & 0 \\ 0 & 0 \\
    \end{array}
      \right] \\
   & + \frac{\omega^{2}}{c^{2}}\left[ \begin{array}{cc}
      (\hat{\epsilon}_{ff}^{LL})^{-1}(\hat{\epsilon}_{ff}^{LT})\hat{\nabla}^{-2}(\hat{\epsilon}_{ff}^{TL})(\hat{\epsilon}_{ff}^{LL})^{-1}  & -(\hat{\epsilon}_{ff}^{LL})^{-1}(\hat{\epsilon}_{ff}^{LT})\hat{\nabla}^{-2} \\ -\hat{\nabla}^{-2}(\hat{\epsilon}_{ff}^{TL})(\hat{\epsilon}_{ff}^{LL})^{-1}
      & \nabla^{-2}\hat{P}^{T}\hat{P}_{f}
      \end{array} \right] + ...,
    \end{split}
\end{equation}
en los casos en que $l << \lambda $, solo nos quedamos con el primer
termino de la expansión y sustituyendolo en \eqref{DielectricFunction} la reescribimos como
\begin{equation}
  \hat{\epsilon}_{M}=\hat{\epsilon}_{aa}-\hat{\epsilon}_{af}(\hat{\epsilon}_{ff}^{LL})^{-1}\hat{\epsilon}_{fa}.
\end{equation}



\section{Implementación}
\section{Resultados}

\begin{thebibliography}{0000}
\bibitem{Metamorphose}
\bibitem{IntroductiontoMetamaterialsandNanophotonics}
\bibitem{ElectromagneticResponseofSystemwithSpatialFluctuations}
\bibitem{Bulkplasmon}
  
  \end{thebibliography}

\end{document}
