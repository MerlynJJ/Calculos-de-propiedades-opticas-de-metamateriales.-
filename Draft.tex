\documentclass[12pt]{article}
\usepackage{graphicx}
\usepackage[spanish]{babel}
\usepackage{float}
\usepackage[utf8]{inputenc}
\usepackage[T1]{fontenc}
\usepackage{outline}
\usepackage{pmgraph}
\usepackage[normalem]{ulem}
\usepackage{amsmath,mathdots}
\usepackage{amsfonts,amssymb,amsthm,yhmath}
\usepackage{fancyhdr}
%\usepackage{multicol}
\usepackage{vmargin}
\usepackage{tikz}
\usepackage{physics}
\usepackage{cancel}
%\usepackage{multicol,multirow}
\usepackage{bm}
\usepackage{hyperref}
\usepackage[scaled]{helvet}
\usepackage{color}
\usepackage{siunitx}
\usepackage{array}
\usepackage{gensymb}
\usepackage{tabularx}
\usepackage{extarrows}
\usepackage{booktabs}
\usepackage[export]{adjustbox}
\newif{\ifignora}
\usetikzlibrary{fadings}
\usetikzlibrary{patterns}
\usetikzlibrary{shadows.blur}
\usetikzlibrary{shapes}

%\renewcommand\familydefault{\sfdefault}
\usepackage{showkeys}
%\usepackage{showlabel}
\hypersetup{
    colorlinks=true,
    linkcolor=black,
    filecolor=magenta,
    urlcolor=blue,
}

\title{Metamateriales \\
  XVIII Escuela de Verano en Física}
\author{ Merlyn Jaqueline Juárez-Gutiérrez y Wolf Luis Mochán\\
  Instituto de Ciencias Físicas}
%\date{}


\begin{document}
\maketitle
\tableofcontents

\section{Introducción}


Un meta-material es un material diseñado artificialmente, formado por dos o más
materiales, con propiedades determinadas por la geometría de sus
componentes y su arreglo. Los meta-materiales se definen como ``{\em Un arreglo de
  elementos estructurales artificiales, diseñados para alcanzar
  propiedades electromagnéticas ventajosas e
  inusuales}''\cite{Metamorphose}. Esta definición es la adoptada en
el {\em Virtual Institute for Artificial Electromagnetic Materials and
  Meta-Materials}.

Las propiedades electromagnéticas de los
meta-materiales están determinadas por sus constituyentes
básicos, a los que se denomina ocasionalmente como meta-átomos, los cuales son
objetos hechos de materiales usuales. Las propiedades de los
meta-materiales pueden ser muy distintas a las de los materiales que
los conforman y pueden llegar a ser muy exóticas. Pueden ser
entonadas escogiendo las formas, estructuras internas, tamaños,
orientaciones mutuas, etc., de sus meta-átomos.  Incluso sus repuestas
individuales pueden ser controladas por señales externas e internas y
por microprocesadores
programables.\cite{IntroductiontoMetamaterialsandNanophotonics}

\subsection{Materiales plasmónicos}
Por ejemplo, si un meta-material tiene componentes metálicos, estos
pueden presentar resonancias asociadas a un movimiento oscilatorio
colectivo de sus electrones de conducción, denominado de acuerdo a
sus características como plasmón de bulto, plasmón de superficie o
plasmón localizados. La frecuencia de este movimiento se
denomina como frecuencia de plasma $\omega_{p}$. Para estimar esta
frecuencia, considere el modelo más simple de un conductor, un gas de
electrones que en equilibrio tiene una densidad de número $n_{0}$,
libre de moverse en un entorno positivamente cargado, de manera que el
sistema sea neutro. Si debido a algún efecto como una
compresión o rarefacción del gas de electrones se produce una
acumulación de carga $Q$, localizada en alguna región $\mathcal{R}$,
esta producirá un campo eléctrico $\vec{E} (\vec{r})$, como ilustra la
figura \ref{Bulkplasmon}. De acuerdo
a la segunda ley de Newton, los electrones adquieren una aceleración
$\frac{d^{2}\vec{r}}{dt^{2}} = -e\vec{E}(\vec{r},t)/m$, donde $m$ y
$-e$ son la masa y la carga eléctrica electrónica. La
velocidad adquirida por los electrones resulta en una corriente
eléctrica  $\vec{j}(\vec{r},t) = -n_0 e \vec v=-n_{0}e({d\vec{r}}/{dt})$
que obedece la ecuación de movimiento
por ${\partial \vec{j}(\vec{r},t)}/{\partial t}=
-\frac{n_{0}e^{2}}{m}\vec{E}(\vec{r},t)$. Integrando la ecuación diferencial de la
corriente sobre una superficie $\Sigma$ que rodee completamente la carga y
usando la ecuación de continuidad y la Ley de Gauss para el campo
eléctrico obtenemos una ecuación diferencial para la carga $Q$
encerrada en $\Sigma$,
\begin{figure}
\centering


\tikzset{every picture/.style={line width=0.75pt}} %set default line width to 0.75pt        

\begin{tikzpicture}[x=0.75pt,y=0.75pt,yscale=-1,xscale=1]
%uncomment if require: \path (0,300); %set diagram left start at 0, and has height of 300

%Shape: Rectangle [id:dp5857108563848268] 
\draw [draw opacity=0][fill={rgb, 255:red, 198; green, 244; blue, 240
  } ,fill opacity=0.48 ] (148,66) -- (390.17,66) -- (390.17,269) --
(148,269) -- cycle ;
%Shape: Circle [id:dp3780902414948827] 
\draw [dash pattern={on 0.84pt off 2.51pt}] (217.04,166.52)
.. controls (217.04,135.56) and (242.14,110.46) .. (273.1,110.46)
.. controls (304.07,110.46) and (329.17,135.56) .. (329.17,166.52)
.. controls (329.17,197.48) and (304.07,222.58) .. (273.1,222.58)
.. controls (242.14,222.58) and (217.04,197.48) .. (217.04,166.52) --
cycle ;
%Shape: Polygon Curved [id:ds008835726195197124] 
\draw [fill={rgb, 255:red, 248; green, 231; blue, 28 } ,fill opacity=1
] (277.08,151.5) .. controls (296.08,166.5) and (293.08,169.5)
.. (269.08,167.5) .. controls (248.08,193.5) and (254.08,143.5)
.. (277.08,151.5) -- cycle ;
%Straight Lines [id:da7163027701373094] 
\draw [color={rgb, 255:red, 0; green, 13; blue, 255 } ,draw opacity=1
] (250.37,181.61) -- (233.46,195.22) ; \draw [shift={(231.9,196.48)},
  rotate = 321.15999999999997] [color={rgb, 255:red, 0; green, 13;
    blue, 255 } ,draw opacity=1 ][line width=0.75] (6.56,-1.97)
.. controls (4.17,-0.84) and (1.99,-0.18) .. (0,0) .. controls
(1.99,0.18) and (4.17,0.84) .. (6.56,1.97) ;
%Straight Lines [id:da27099612271830587] 
\draw [color={rgb, 255:red, 65; green, 117; blue, 5 } ,draw opacity=1
] (256.82,188.49) -- (247.21,198.4) ; \draw [shift={(245.82,199.84)},
  rotate = 314.11] [color={rgb, 255:red, 65; green, 117; blue, 5 }
  ,draw opacity=1 ][line width=0.75] (6.56,-1.97) .. controls
(4.17,-0.84) and (1.99,-0.18) .. (0,0) .. controls (1.99,0.18) and
(4.17,0.84) .. (6.56,1.97) ;
%Straight Lines [id:da882164603438807] 
\draw [color={rgb, 255:red, 4; green, 0; blue, 255 } ,draw opacity=1 ]
(290.5,185.75) -- (306.35,200.58) ; \draw [shift={(307.81,201.95)},
  rotate = 223.09] [color={rgb, 255:red, 4; green, 0; blue, 255 }
  ,draw opacity=1 ][line width=0.75] (6.56,-1.97) .. controls
(4.17,-0.84) and (1.99,-0.18) .. (0,0) .. controls (1.99,0.18) and
(4.17,0.84) .. (6.56,1.97) ;
%Straight Lines [id:da884690742053359] 
\draw [color={rgb, 255:red, 65; green, 117; blue, 5 } ,draw opacity=1
] (296.41,178.39) -- (307.57,186.52) ; \draw [shift={(309.19,187.7)},
  rotate = 216.04] [color={rgb, 255:red, 65; green, 117; blue, 5 }
  ,draw opacity=1 ][line width=0.75] (6.56,-1.97) .. controls
(4.17,-0.84) and (1.99,-0.18) .. (0,0) .. controls (1.99,0.18) and
(4.17,0.84) .. (6.56,1.97) ;
%Straight Lines [id:da20074876184087243] 
\draw [color={rgb, 255:red, 41; green, 51; blue, 241 } ,draw opacity=1
] (270.3,191.49) -- (270.4,213.19) ; \draw [shift={(270.41,215.19)},
  rotate = 269.72] [color={rgb, 255:red, 41; green, 51; blue, 241 }
  ,draw opacity=1 ][line width=0.75] (6.56,-1.97) .. controls
(4.17,-0.84) and (1.99,-0.18) .. (0,0) .. controls (1.99,0.18) and
(4.17,0.84) .. (6.56,1.97) ;
%Straight Lines [id:da608342686522136] 
\draw [color={rgb, 255:red, 65; green, 117; blue, 5 } ,draw opacity=1
] (279.7,190.73) -- (281.46,204.42) ; \draw [shift={(281.72,206.41)},
  rotate = 262.67] [color={rgb, 255:red, 65; green, 117; blue, 5 }
  ,draw opacity=1 ][line width=0.75] (6.56,-1.97) .. controls
(4.17,-0.84) and (1.99,-0.18) .. (0,0) .. controls (1.99,0.18) and
(4.17,0.84) .. (6.56,1.97) ;
%Straight Lines [id:da088624174223913] 
\draw [color={rgb, 255:red, 4; green, 0; blue, 255 } ,draw opacity=1 ]
(247.34,162.12) -- (225.65,161.27) ; \draw [shift={(223.65,161.19)},
  rotate = 362.25] [color={rgb, 255:red, 4; green, 0; blue, 255 }
  ,draw opacity=1 ][line width=0.75] (6.56,-1.97) .. controls
(4.17,-0.84) and (1.99,-0.18) .. (0,0) .. controls (1.99,0.18) and
(4.17,0.84) .. (6.56,1.97) ;
%Straight Lines [id:da3058655897728151] 
\draw [color={rgb, 255:red, 65; green, 117; blue, 5 } ,draw opacity=1
] (247.69,171.55) -- (233.92,172.7) ; \draw [shift={(231.93,172.87)},
  rotate = 355.2] [color={rgb, 255:red, 65; green, 117; blue, 5 }
  ,draw opacity=1 ][line width=0.75] (6.56,-1.97) .. controls
(4.17,-0.84) and (1.99,-0.18) .. (0,0) .. controls (1.99,0.18) and
(4.17,0.84) .. (6.56,1.97) ;
%Straight Lines [id:da7992511633354091] 
\draw [color={rgb, 255:red, 4; green, 0; blue, 255 } ,draw opacity=1 ]
(296.59,151.57) -- (313.23,137.63) ; \draw [shift={(314.76,136.34)},
  rotate = 500.02] [color={rgb, 255:red, 4; green, 0; blue, 255 }
  ,draw opacity=1 ][line width=0.75] (6.56,-1.97) .. controls
(4.17,-0.84) and (1.99,-0.18) .. (0,0) .. controls (1.99,0.18) and
(4.17,0.84) .. (6.56,1.97) ;
%Straight Lines [id:da021188728886464392] 
\draw [color={rgb, 255:red, 65; green, 117; blue, 5 } ,draw opacity=1
] (290,144.82) -- (299.42,134.72) ; \draw [shift={(300.78,133.26)},
  rotate = 492.97] [color={rgb, 255:red, 65; green, 117; blue, 5 }
  ,draw opacity=1 ][line width=0.75] (6.56,-1.97) .. controls
(4.17,-0.84) and (1.99,-0.18) .. (0,0) .. controls (1.99,0.18) and
(4.17,0.84) .. (6.56,1.97) ;
%Straight Lines [id:da5888482352950647] 
\draw [color={rgb, 255:red, 4; green, 0; blue, 255 } ,draw opacity=1 ]
(254.98,145.05) -- (240.02,129.33) ; \draw [shift={(238.64,127.88)},
  rotate = 406.4] [color={rgb, 255:red, 4; green, 0; blue, 255 } ,draw
  opacity=1 ][line width=0.75] (6.56,-1.97) .. controls (4.17,-0.84)
and (1.99,-0.18) .. (0,0) .. controls (1.99,0.18) and (4.17,0.84)
.. (6.56,1.97) ;
%Straight Lines [id:da4986041081219512] 
\draw [color={rgb, 255:red, 65; green, 117; blue, 5 } ,draw opacity=1
] (248.67,152.06) -- (237.99,143.3) ; \draw [shift={(236.44,142.03)},
  rotate = 399.35] [color={rgb, 255:red, 65; green, 117; blue, 5 }
  ,draw opacity=1 ][line width=0.75] (6.56,-1.97) .. controls
(4.17,-0.84) and (1.99,-0.18) .. (0,0) .. controls (1.99,0.18) and
(4.17,0.84) .. (6.56,1.97) ;
%Straight Lines [id:da8801881465271447] 
\draw [color={rgb, 255:red, 0; green, 13; blue, 255 } ,draw opacity=1
] (276.23,135.96) -- (277.24,114.27) ; \draw [shift={(277.33,112.28)},
  rotate = 452.65] [color={rgb, 255:red, 0; green, 13; blue, 255 }
  ,draw opacity=1 ][line width=0.75] (6.56,-1.97) .. controls
(4.17,-0.84) and (1.99,-0.18) .. (0,0) .. controls (1.99,0.18) and
(4.17,0.84) .. (6.56,1.97) ;
%Straight Lines [id:da03421807598256077] 
\draw [color={rgb, 255:red, 65; green, 117; blue, 5 } ,draw opacity=1
] (266.8,136.24) -- (265.75,122.47) ; \draw [shift={(265.59,120.47)},
  rotate = 445.6] [color={rgb, 255:red, 65; green, 117; blue, 5 }
  ,draw opacity=1 ][line width=0.75] (6.56,-1.97) .. controls
(4.17,-0.84) and (1.99,-0.18) .. (0,0) .. controls (1.99,0.18) and
(4.17,0.84) .. (6.56,1.97) ;
%Straight Lines [id:da42029107361898876] 
\draw [color={rgb, 255:red, 4; green, 0; blue, 255 } ,draw opacity=1 ]
(297.28,170.57) -- (318.97,171.32) ; \draw [shift={(320.97,171.39)},
  rotate = 181.97] [color={rgb, 255:red, 4; green, 0; blue, 255 }
  ,draw opacity=1 ][line width=0.75] (6.56,-1.97) .. controls
(4.17,-0.84) and (1.99,-0.18) .. (0,0) .. controls (1.99,0.18) and
(4.17,0.84) .. (6.56,1.97) ;
%Straight Lines [id:da34990856882657484] 
\draw [color={rgb, 255:red, 65; green, 117; blue, 5 } ,draw opacity=1
] (296.89,161.15) -- (310.64,159.92) ; \draw [shift={(312.64,159.75)},
  rotate = 534.9200000000001] [color={rgb, 255:red, 65; green, 117;
    blue, 5 } ,draw opacity=1 ][line width=0.75] (6.56,-1.97)
.. controls (4.17,-0.84) and (1.99,-0.18) .. (0,0) .. controls
(1.99,0.18) and (4.17,0.84) .. (6.56,1.97) ;
%Straight Lines [id:da8568463836228528] 
\draw [color={rgb, 255:red, 0; green, 13; blue, 255 } ,draw opacity=1
] (219.17,206) -- (205.78,215.82) ; \draw [shift={(204.17,217)},
  rotate = 323.75] [color={rgb, 255:red, 0; green, 13; blue, 255 }
  ,draw opacity=1 ][line width=0.75] (6.56,-1.97) .. controls
(4.17,-0.84) and (1.99,-0.18) .. (0,0) .. controls (1.99,0.18) and
(4.17,0.84) .. (6.56,1.97) ;
%Straight Lines [id:da6974865872969185] 
\draw [color={rgb, 255:red, 65; green, 117; blue, 5 } ,draw opacity=1
] (233.08,216.87) -- (226.48,224.49) ; \draw [shift={(225.17,226)},
  rotate = 310.94] [color={rgb, 255:red, 65; green, 117; blue, 5 }
  ,draw opacity=1 ][line width=0.75] (6.56,-1.97) .. controls
(4.17,-0.84) and (1.99,-0.18) .. (0,0) .. controls (1.99,0.18) and
(4.17,0.84) .. (6.56,1.97) ;
%Straight Lines [id:da9473192312247024] 
\draw [color={rgb, 255:red, 4; green, 0; blue, 255 } ,draw opacity=1 ]
(319.86,212.07) -- (329.76,222.08) ; \draw [shift={(331.17,223.5)},
  rotate = 225.31] [color={rgb, 255:red, 4; green, 0; blue, 255 }
  ,draw opacity=1 ][line width=0.75] (6.56,-1.97) .. controls
(4.17,-0.84) and (1.99,-0.18) .. (0,0) .. controls (1.99,0.18) and
(4.17,0.84) .. (6.56,1.97) ;
%Straight Lines [id:da5546786921432448] 
\draw [color={rgb, 255:red, 65; green, 117; blue, 5 } ,draw opacity=1
] (330.42,198.47) -- (338.39,202.58) ; \draw [shift={(340.17,203.5)},
  rotate = 207.31] [color={rgb, 255:red, 65; green, 117; blue, 5 }
  ,draw opacity=1 ][line width=0.75] (6.56,-1.97) .. controls
(4.17,-0.84) and (1.99,-0.18) .. (0,0) .. controls (1.99,0.18) and
(4.17,0.84) .. (6.56,1.97) ;
%Straight Lines [id:da2980392776012871] 
\draw [color={rgb, 255:red, 4; green, 0; blue, 255 } ,draw opacity=1 ]
(208.09,160.96) -- (192.16,160.11) ; \draw [shift={(190.17,160)},
  rotate = 363.07] [color={rgb, 255:red, 4; green, 0; blue, 255 }
  ,draw opacity=1 ][line width=0.75] (6.56,-1.97) .. controls
(4.17,-0.84) and (1.99,-0.18) .. (0,0) .. controls (1.99,0.18) and
(4.17,0.84) .. (6.56,1.97) ;
%Straight Lines [id:da3672184968558653] 
\draw [color={rgb, 255:red, 65; green, 117; blue, 5 } ,draw opacity=1
] (208,173.95) -- (198.17,173.99) ; \draw [shift={(196.17,174)},
  rotate = 359.77] [color={rgb, 255:red, 65; green, 117; blue, 5 }
  ,draw opacity=1 ][line width=0.75] (6.56,-1.97) .. controls
(4.17,-0.84) and (1.99,-0.18) .. (0,0) .. controls (1.99,0.18) and
(4.17,0.84) .. (6.56,1.97) ;
%Straight Lines [id:da11017355529948814] 
\draw [color={rgb, 255:red, 4; green, 0; blue, 255 } ,draw opacity=1 ]
(325.09,129.6) -- (335.44,123.51) ; \draw [shift={(337.17,122.5)},
  rotate = 509.55] [color={rgb, 255:red, 4; green, 0; blue, 255 }
  ,draw opacity=1 ][line width=0.75] (6.56,-1.97) .. controls
(4.17,-0.84) and (1.99,-0.18) .. (0,0) .. controls (1.99,0.18) and
(4.17,0.84) .. (6.56,1.97) ;
%Straight Lines [id:da6955212381544827] 
\draw [color={rgb, 255:red, 65; green, 117; blue, 5 } ,draw opacity=1
] (312.53,116.67) -- (319.71,109.87) ; \draw [shift={(321.17,108.5)},
  rotate = 496.58] [color={rgb, 255:red, 65; green, 117; blue, 5 }
  ,draw opacity=1 ][line width=0.75] (6.56,-1.97) .. controls
(4.17,-0.84) and (1.99,-0.18) .. (0,0) .. controls (1.99,0.18) and
(4.17,0.84) .. (6.56,1.97) ;
%Straight Lines [id:da3603103334401869] 
\draw [color={rgb, 255:red, 4; green, 0; blue, 255 } ,draw opacity=1 ]
(232.17,118) -- (218.49,102.5) ; \draw [shift={(217.17,101)}, rotate =
  408.58000000000004] [color={rgb, 255:red, 4; green, 0; blue, 255 }
  ,draw opacity=1 ][line width=0.75] (6.56,-1.97) .. controls
(4.17,-0.84) and (1.99,-0.18) .. (0,0) .. controls (1.99,0.18) and
(4.17,0.84) .. (6.56,1.97) ;
%Straight Lines [id:da6369305975573354] 
\draw [color={rgb, 255:red, 65; green, 117; blue, 5 } ,draw opacity=1
] (220.59,128.62) -- (212.8,123.15) ; \draw [shift={(211.17,122)},
  rotate = 395.12] [color={rgb, 255:red, 65; green, 117; blue, 5 }
  ,draw opacity=1 ][line width=0.75] (6.56,-1.97) .. controls
(4.17,-0.84) and (1.99,-0.18) .. (0,0) .. controls (1.99,0.18) and
(4.17,0.84) .. (6.56,1.97) ;
%Straight Lines [id:da3610436111733121] 
\draw [color={rgb, 255:red, 0; green, 13; blue, 255 } ,draw opacity=1
] (278.17,102) -- (279.07,84) ; \draw [shift={(279.17,82)}, rotate =
  452.86] [color={rgb, 255:red, 0; green, 13; blue, 255 } ,draw
  opacity=1 ][line width=0.75] (6.56,-1.97) .. controls (4.17,-0.84)
and (1.99,-0.18) .. (0,0) .. controls (1.99,0.18) and (4.17,0.84)
.. (6.56,1.97) ;
%Straight Lines [id:da8453237207742365] 
\draw [color={rgb, 255:red, 65; green, 117; blue, 5 } ,draw opacity=1
] (263.17,103) -- (262.39,95.99) ; \draw [shift={(262.17,94)}, rotate
  = 443.66] [color={rgb, 255:red, 65; green, 117; blue, 5 } ,draw
  opacity=1 ][line width=0.75] (6.56,-1.97) .. controls (4.17,-0.84)
and (1.99,-0.18) .. (0,0) .. controls (1.99,0.18) and (4.17,0.84)
.. (6.56,1.97) ;
%Straight Lines [id:da8500846118815689] 
\draw [color={rgb, 255:red, 4; green, 0; blue, 255 } ,draw opacity=1 ]
(335.91,171.88) -- (347.17,172.41) ; \draw [shift={(349.17,172.5)},
  rotate = 182.66] [color={rgb, 255:red, 4; green, 0; blue, 255 }
  ,draw opacity=1 ][line width=0.75] (6.56,-1.97) .. controls
(4.17,-0.84) and (1.99,-0.18) .. (0,0) .. controls (1.99,0.18) and
(4.17,0.84) .. (6.56,1.97) ;
%Straight Lines [id:da35490233668558013] 
\draw [color={rgb, 255:red, 65; green, 117; blue, 5 } ,draw opacity=1
] (337.69,158.9) -- (345.19,157.79) ; \draw [shift={(347.17,157.5)},
  rotate = 531.61] [color={rgb, 255:red, 65; green, 117; blue, 5 }
  ,draw opacity=1 ][line width=0.75] (6.56,-1.97) .. controls
(4.17,-0.84) and (1.99,-0.18) .. (0,0) .. controls (1.99,0.18) and
(4.17,0.84) .. (6.56,1.97) ;
%Straight Lines [id:da09061677521708489] 
\draw [color={rgb, 255:red, 4; green, 0; blue, 255 } ,draw opacity=1 ]
(270.17,232) -- (269.63,246.1) ; \draw [shift={(269.55,248.1)}, rotate
  = 272.19] [color={rgb, 255:red, 4; green, 0; blue, 255 } ,draw
  opacity=1 ][line width=0.75] (6.56,-1.97) .. controls (4.17,-0.84)
and (1.99,-0.18) .. (0,0) .. controls (1.99,0.18) and (4.17,0.84)
.. (6.56,1.97) ;
%Straight Lines [id:da33913903399046463] 
\draw [color={rgb, 255:red, 65; green, 117; blue, 5 } ,draw opacity=1
] (285.64,228.91) -- (287.72,238.05) ; \draw [shift={(288.17,240)},
  rotate = 257.17] [color={rgb, 255:red, 65; green, 117; blue, 5 }
  ,draw opacity=1 ][line width=0.75] (6.56,-1.97) .. controls
(4.17,-0.84) and (1.99,-0.18) .. (0,0) .. controls (1.99,0.18) and
(4.17,0.84) .. (6.56,1.97) ;


% Text Node
\draw (204,138.4) node [anchor=north west][inner sep=0.75pt] {$\Sigma
  $};
% Text Node
\draw (273.23,157.14) node  [font=\scriptsize]  {$\boldsymbol{Q}$};


\end{tikzpicture}

\caption{Acumulación de carga $Q$ en cierta región
  $\mathcal{R}$ dentro de un conductor homogéneo, rodeada
  por una superficie $\Sigma$ sobre la cual se aplica la ley
  de Gauss al campo producido por $Q$, campo que produce una densidad
  de corriente que fluye a través de $\Sigma$. La corriente
  modifica $Q$ conforme transcurre el tiempo $t$ produciendo las
  llamadas oscilaciones de plasma.}
\label{Bulkplasmon}
\end{figure}
\begin{equation}
  \label{ChargeDifEq}
  \frac{d^{2}Q}{dt^{2}}=-\frac{4\pi n_0 e^{2}}{m}Q,
\end{equation}
la cual es una ecuación diferencial idéntica a la de un
oscilador armónico simple como el que se ilustra en la fig.
\ref{OscArmonicoSimple}:
\begin{figure}
  \centering
  
\begin{tikzpicture}[x=0.9pt,y=0.9pt,yscale=-1,xscale=1]
%uncomment if require: \path (0,193); %set diagram left start at 0,
%and has height of 193

%Shape: Inductor (Air Core) [id:dp34048340969550106]
\draw (119.21,42.46) -- (119.25,59.21) .. controls (131.85,59.42) and
(142.62,61.72) .. (146.39,65.01) .. controls (150.17,68.29) and
(146.17,71.88) .. (136.32,74.06) .. controls (128.64,75.74) and
(118.71,76.44) .. (109.06,75.98) .. controls (105.3,75.99) and
(102.24,75.16) .. (102.24,74.14) .. controls (102.24,73.11) and
(105.29,72.27) .. (109.05,72.26) .. controls (118.7,71.76) and
(128.63,72.42) .. (136.32,74.06) .. controls (144.52,75.98) and
(149.16,78.66) .. (149.17,81.48) .. controls (149.18,84.29) and
(144.54,87) .. (136.36,88.95) .. controls (128.68,90.63) and
(118.74,91.33) .. (109.09,90.87) .. controls (105.33,90.88) and
(102.27,90.05) .. (102.27,89.03) .. controls (102.27,88) and
(105.32,87.16) .. (109.08,87.15) .. controls (118.73,86.65) and
(128.67,87.3) .. (136.36,88.95) .. controls (144.55,90.86) and
(149.2,93.55) .. (149.2,96.36) .. controls (149.21,99.18) and
(144.57,101.88) .. (136.39,103.84) .. controls (128.71,105.52) and
(118.77,106.22) .. (109.13,105.76) .. controls (105.36,105.77) and
(102.31,104.94) .. (102.31,103.91) .. controls (102.3,102.89) and
(105.36,102.05) .. (109.12,102.04) .. controls (118.76,101.54) and
(128.7,102.19) .. (136.39,103.84) .. controls (146.25,105.97) and
(150.26,109.55) .. (146.5,112.85) .. controls (142.74,116.15) and
(131.98,118.49) .. (119.38,118.76) -- (119.42,135.51) ;
%Shape: Rectangle [id:dp47577099884182095]
\draw [fill={rgb, 255:red, 245; green, 166; blue, 35 } ,fill opacity=1
] (87,27.5) -- (151.17,27.5) -- (151.17,43) -- (87,43) -- cycle ;
%Shape: Circle [id:dp24543618839132486]
\draw [fill={rgb, 255:red, 126; green, 211; blue, 33 } ,fill opacity=1
] (109.67,145.26) .. controls (109.67,139.88) and (114.04,135.51)
.. (119.42,135.51) .. controls (124.8,135.51) and (129.17,139.88)
.. (129.17,145.26) .. controls (129.17,150.65) and (124.8,155.01)
.. (119.42,155.01) .. controls (114.04,155.01) and (109.67,150.65)
.. (109.67,145.26) -- cycle ;
%Shape: Inductor (Air Core) [id:dp8275323218505621]
\draw (42.21,42.46) -- (42.23,51.56) .. controls (54.83,51.66) and
(65.6,52.9) .. (69.37,54.68) .. controls (73.14,56.46) and
(69.14,58.41) .. (59.29,59.61) .. controls (51.61,60.53) and
(41.67,60.92) .. (32.02,60.68) .. controls (28.26,60.69) and
(25.21,60.24) .. (25.21,59.68) .. controls (25.2,59.13) and
(28.26,58.67) .. (32.02,58.66) .. controls (41.67,58.37) and
(51.6,58.72) .. (59.29,59.61) .. controls (67.48,60.64) and
(72.12,62.09) .. (72.13,63.62) .. controls (72.13,65.15) and
(67.49,66.62) .. (59.31,67.69) .. controls (51.63,68.61) and
(41.69,69) .. (32.04,68.76) .. controls (28.28,68.77) and
(25.23,68.33) .. (25.22,67.77) .. controls (25.22,67.21) and
(28.28,66.75) .. (32.04,66.74) .. controls (41.68,66.46) and
(51.62,66.81) .. (59.31,67.69) .. controls (67.5,68.72) and
(72.14,70.18) .. (72.15,71.71) .. controls (72.15,73.24) and
(67.51,74.71) .. (59.33,75.78) .. controls (51.64,76.7) and
(41.71,77.09) .. (32.06,76.85) .. controls (28.3,76.86) and
(25.24,76.41) .. (25.24,75.86) .. controls (25.24,75.3) and
(28.29,74.84) .. (32.06,74.83) .. controls (41.7,74.55) and
(51.64,74.89) .. (59.33,75.78) .. controls (69.18,76.93) and
(73.19,78.86) .. (69.43,80.66) .. controls (65.67,82.46) and
(54.9,83.74) .. (42.3,83.9) -- (42.32,93) ;
%Shape: Rectangle [id:dp7392561467625306]
\draw [fill={rgb, 255:red, 245; green, 166; blue, 35 } ,fill opacity=1
] (10,27.5) -- (74.17,27.5) -- (74.17,43) -- (10,43) -- cycle ;
%Shape: Circle [id:dp15959587945734588]
\draw [fill={rgb, 255:red, 126; green, 211; blue, 33 } ,fill opacity=1
] (32.57,102.75) .. controls (32.57,97.36) and (36.94,93)
.. (42.32,93) .. controls (47.71,93) and (52.07,97.36)
.. (52.07,102.75) .. controls (52.07,108.13) and (47.71,112.5)
.. (42.32,112.5) .. controls (36.94,112.5) and (32.57,108.13)
.. (32.57,102.75) -- cycle ;

%Straight Lines [id:da8898992399758622]
\draw (95.32,102.75) -- (95.17,147) ; \draw [shift={(95.17,147)},
  rotate = 270.2] [color={rgb, 255:red, 0; green, 0; blue, 0 } ][line
  width=0.75] (0,5.59) -- (0,-5.59) ; \draw [shift={(95.32,102.75)},
  rotate = 270.2] [color={rgb, 255:red, 0; green, 0; blue, 0 } ][line
  width=0.75] (0,5.59) -- (0,-5.59) ;


% Text Node
\draw (5,58.4) node [anchor=north west][inner sep=0.75pt]    {$k$};
% Text Node
\draw (57,100.4) node [anchor=north west][inner sep=0.75pt]    {$m$};
% Text Node
\draw (132,134.4) node [anchor=north west][inner sep=0.75pt]    {$m$};
% Text Node
\draw (83,113.4) node [anchor=north west][inner sep=0.75pt]    {$y$};


\end{tikzpicture}

  \caption{\label{OscArmonicoSimple}
    Oscilador armónico de masa $m$ y constante $k$ en su posición de
    equilibrio y desplazado una distancia $y$.}
\end{figure}
Sustituyendo la Ley de Hooke $F= -ky$ en la segunda ley
de Newton $ma=F$ para un oscilador con constante $k$ y masa $m$ obtenemos
\begin{equation}
  \label{OAEq}
  \frac{d^{2}y}{dt^{2}}=-\omega^2y,
\end{equation}
donde $\omega=\sqrt{k/m}$, y cuya solución, $y(t)=y_{0}\cos(\omega t+\phi_0)$
es un movimiento periódico con una frecuencia $\omega$ que depende de
$k$ y $m$.
Comparando las ecs.  y \eqref{ChargeDifEq} y \eqref{OAEq} notamos que
puede existir carga en el seno de nuestro metal modelo, pero ésta
oscila con la {\em frecuencia de plasma} $\omega_p$ dada por
\begin{equation}
  \label{plasmafec}
  \omega _{p}^{2} = \frac{4\pi n_0e^{2}}{m}.
\end{equation}
La repulsión mutua entre electrones los impulsa lejos
de regiones en que haya una densidad electrónica excedente, por arriba
de su valor nominal. El movimiento consecuente prosigue aun después de
que el sistema se neutralice debido a la inercia electrónica, que los
hará proseguir su camino hasta que en la región orignal disminuya
tanto la densidad de electrones que aparezca una carga neta positiva
que frene a los electrones en fuga y los haga regresar, hasta que su repulsión
mutua los frene al haber regresado a la configuración inicial. Este
proceso se repetie periódicamente y su frecuencia $\omega_p$ está
relacionada con la repulsión coulombiana, proporcional a $e^2$, la
densidad de de número electrónica $n_0$ y la inercia electrónica
caracterizada por $m$.

En lugar de un medio infinito, consideremos ahora un medio
semiinfinito separado del vacío por una superficie plana.
Un análisis análogo nos permite obtener la frecuencia del {\em plasmón de
superficie}, considerando ahora un exceso de carga $Q$ en una región $R$ en
la interfaz, como ilustra la fig.
\ref{Surfplasmon}, el cual produce un campo eléctrico $\vec{E}(\vec{r},t)$
que induce corrientes en el conductor.
\begin{figure}
  \centering
  \tikzset{every picture/.style={line width=0.75pt}} %set default line width to 0.75pt        

\begin{tikzpicture}[x=0.75pt,y=0.75pt,yscale=-1,xscale=1]
%uncomment if require: \path (0,300); %set diagram left start at 0,
%and has height of 300

%Shape: Rectangle [id:dp7956545958448051] 
\draw [draw opacity=0][fill={rgb, 255:red, 198; green, 244; blue, 240
  } ,fill opacity=0.48 ] (114.02,144) -- (356.19,144) --
(356.19,248.02) -- (114.02,248.02) -- cycle ;
%Shape: Circle [id:dp44476661130132666] 
\draw [dash pattern={on 0.84pt off 2.51pt}] (180.04,142.52)
.. controls (180.04,111.56) and (205.14,86.46) .. (236.1,86.46)
.. controls (267.07,86.46) and (292.17,111.56) .. (292.17,142.52)
.. controls (292.17,173.48) and (267.07,198.58) .. (236.1,198.58)
.. controls (205.14,198.58) and (180.04,173.48) .. (180.04,142.52) --
cycle ;
%Straight Lines [id:da14033928295269626] 
\draw [color={rgb, 255:red, 0; green, 13; blue, 255 } ,draw opacity=1
] (213.37,157.61) -- (196.46,171.22) ; \draw [shift={(194.9,172.48)},
  rotate = 321.15999999999997] [color={rgb, 255:red, 0; green, 13;
    blue, 255 } ,draw opacity=1 ][line width=0.75] (6.56,-1.97)
.. controls (4.17,-0.84) and (1.99,-0.18) .. (0,0) .. controls
(1.99,0.18) and (4.17,0.84) .. (6.56,1.97) ;
%Straight Lines [id:da45124177831622647] 
\draw [color={rgb, 255:red, 65; green, 117; blue, 5 } ,draw opacity=1
] (219.82,164.49) -- (210.21,174.4) ; \draw [shift={(208.82,175.84)},
  rotate = 314.11] [color={rgb, 255:red, 65; green, 117; blue, 5 }
  ,draw opacity=1 ][line width=0.75] (6.56,-1.97) .. controls
(4.17,-0.84) and (1.99,-0.18) .. (0,0) .. controls (1.99,0.18) and
(4.17,0.84) .. (6.56,1.97) ;
%Straight Lines [id:da044509622606934585] 
\draw [color={rgb, 255:red, 4; green, 0; blue, 255 } ,draw opacity=1 ]
(253.5,161.75) -- (269.35,176.58) ; \draw [shift={(270.81,177.95)},
  rotate = 223.09] [color={rgb, 255:red, 4; green, 0; blue, 255 }
  ,draw opacity=1 ][line width=0.75] (6.56,-1.97) .. controls
(4.17,-0.84) and (1.99,-0.18) .. (0,0) .. controls (1.99,0.18) and
(4.17,0.84) .. (6.56,1.97) ;
%Straight Lines [id:da16284278441076816] 
\draw [color={rgb, 255:red, 65; green, 117; blue, 5 } ,draw opacity=1
] (259.41,154.39) -- (270.57,162.52) ; \draw [shift={(272.19,163.7)},
  rotate = 216.04] [color={rgb, 255:red, 65; green, 117; blue, 5 }
  ,draw opacity=1 ][line width=0.75] (6.56,-1.97) .. controls
(4.17,-0.84) and (1.99,-0.18) .. (0,0) .. controls (1.99,0.18) and
(4.17,0.84) .. (6.56,1.97) ;
%Straight Lines [id:da8217139032715733] 
\draw [color={rgb, 255:red, 41; green, 51; blue, 241 } ,draw opacity=1
] (233.3,167.49) -- (233.4,189.19) ; \draw [shift={(233.41,191.19)},
  rotate = 269.72] [color={rgb, 255:red, 41; green, 51; blue, 241 }
  ,draw opacity=1 ][line width=0.75] (6.56,-1.97) .. controls
(4.17,-0.84) and (1.99,-0.18) .. (0,0) .. controls (1.99,0.18) and
(4.17,0.84) .. (6.56,1.97) ;
%Straight Lines [id:da4291710065118748] 
\draw [color={rgb, 255:red, 65; green, 117; blue, 5 } ,draw opacity=1
] (242.7,166.73) -- (244.46,180.42) ; \draw [shift={(244.72,182.41)},
  rotate = 262.67] [color={rgb, 255:red, 65; green, 117; blue, 5 }
  ,draw opacity=1 ][line width=0.75] (6.56,-1.97) .. controls
(4.17,-0.84) and (1.99,-0.18) .. (0,0) .. controls (1.99,0.18) and
(4.17,0.84) .. (6.56,1.97) ;
%Straight Lines [id:da8972267016384223] 
\draw [color={rgb, 255:red, 4; green, 0; blue, 255 } ,draw opacity=1 ]
(210.34,138.12) -- (188.65,137.27) ; \draw [shift={(186.65,137.19)},
  rotate = 362.25] [color={rgb, 255:red, 4; green, 0; blue, 255 }
  ,draw opacity=1 ][line width=0.75] (6.56,-1.97) .. controls
(4.17,-0.84) and (1.99,-0.18) .. (0,0) .. controls (1.99,0.18) and
(4.17,0.84) .. (6.56,1.97) ;
%Straight Lines [id:da813137464274293] 
\draw [color={rgb, 255:red, 65; green, 117; blue, 5 } ,draw opacity=1
] (210.69,147.55) -- (196.92,148.7) ; \draw [shift={(194.93,148.87)},
  rotate = 355.2] [color={rgb, 255:red, 65; green, 117; blue, 5 }
  ,draw opacity=1 ][line width=0.75] (6.56,-1.97) .. controls
(4.17,-0.84) and (1.99,-0.18) .. (0,0) .. controls (1.99,0.18) and
(4.17,0.84) .. (6.56,1.97) ;
%Straight Lines [id:da3182154012798515] 
\draw [color={rgb, 255:red, 4; green, 0; blue, 255 } ,draw opacity=1 ]
(259.59,127.57) -- (276.23,113.63) ; \draw [shift={(277.76,112.34)},
  rotate = 500.02] [color={rgb, 255:red, 4; green, 0; blue, 255 }
  ,draw opacity=1 ][line width=0.75] (6.56,-1.97) .. controls
(4.17,-0.84) and (1.99,-0.18) .. (0,0) .. controls (1.99,0.18) and
(4.17,0.84) .. (6.56,1.97) ;
%Straight Lines [id:da18228076317796438] 
\draw [color={rgb, 255:red, 65; green, 117; blue, 5 } ,draw opacity=1
] (253,120.82) -- (262.42,110.72) ; \draw [shift={(263.78,109.26)},
  rotate = 492.97] [color={rgb, 255:red, 65; green, 117; blue, 5 }
  ,draw opacity=1 ][line width=0.75] (6.56,-1.97) .. controls
(4.17,-0.84) and (1.99,-0.18) .. (0,0) .. controls (1.99,0.18) and
(4.17,0.84) .. (6.56,1.97) ;
%Straight Lines [id:da6141206217382678] 
\draw [color={rgb, 255:red, 4; green, 0; blue, 255 } ,draw opacity=1 ]
(217.98,121.05) -- (203.02,105.33) ; \draw [shift={(201.64,103.88)},
  rotate = 406.4] [color={rgb, 255:red, 4; green, 0; blue, 255 } ,draw
  opacity=1 ][line width=0.75] (6.56,-1.97) .. controls (4.17,-0.84)
and (1.99,-0.18) .. (0,0) .. controls (1.99,0.18) and (4.17,0.84)
.. (6.56,1.97) ;
%Straight Lines [id:da8166253971322168] 
\draw [color={rgb, 255:red, 65; green, 117; blue, 5 } ,draw opacity=1
] (211.67,128.06) -- (200.99,119.3) ; \draw [shift={(199.44,118.03)},
  rotate = 399.35] [color={rgb, 255:red, 65; green, 117; blue, 5 }
  ,draw opacity=1 ][line width=0.75] (6.56,-1.97) .. controls
(4.17,-0.84) and (1.99,-0.18) .. (0,0) .. controls (1.99,0.18) and
(4.17,0.84) .. (6.56,1.97) ;
%Straight Lines [id:da3465118382665756] 
\draw [color={rgb, 255:red, 0; green, 13; blue, 255 } ,draw opacity=1
] (239.23,111.96) -- (240.24,90.27) ; \draw [shift={(240.33,88.28)},
  rotate = 452.65] [color={rgb, 255:red, 0; green, 13; blue, 255 }
  ,draw opacity=1 ][line width=0.75] (6.56,-1.97) .. controls
(4.17,-0.84) and (1.99,-0.18) .. (0,0) .. controls (1.99,0.18) and
(4.17,0.84) .. (6.56,1.97) ;
%Straight Lines [id:da9358764629291445] 
\draw [color={rgb, 255:red, 65; green, 117; blue, 5 } ,draw opacity=1
] (229.8,112.24) -- (228.75,98.47) ; \draw [shift={(228.59,96.47)},
  rotate = 445.6] [color={rgb, 255:red, 65; green, 117; blue, 5 }
  ,draw opacity=1 ][line width=0.75] (6.56,-1.97) .. controls
(4.17,-0.84) and (1.99,-0.18) .. (0,0) .. controls (1.99,0.18) and
(4.17,0.84) .. (6.56,1.97) ;
%Straight Lines [id:da5941951041547017] 
\draw [color={rgb, 255:red, 4; green, 0; blue, 255 } ,draw opacity=1 ]
(260.28,146.57) -- (281.97,147.32) ; \draw [shift={(283.97,147.39)},
  rotate = 181.97] [color={rgb, 255:red, 4; green, 0; blue, 255 }
  ,draw opacity=1 ][line width=0.75] (6.56,-1.97) .. controls
(4.17,-0.84) and (1.99,-0.18) .. (0,0) .. controls (1.99,0.18) and
(4.17,0.84) .. (6.56,1.97) ;
%Straight Lines [id:da9395957665484183] 
\draw [color={rgb, 255:red, 65; green, 117; blue, 5 } ,draw opacity=1
] (259.89,137.15) -- (273.64,135.92) ; \draw [shift={(275.64,135.75)},
  rotate = 534.9200000000001] [color={rgb, 255:red, 65; green, 117;
    blue, 5 } ,draw opacity=1 ][line width=0.75] (6.56,-1.97)
.. controls (4.17,-0.84) and (1.99,-0.18) .. (0,0) .. controls
(1.99,0.18) and (4.17,0.84) .. (6.56,1.97) ;
%Straight Lines [id:da6754579892739427] 
\draw [color={rgb, 255:red, 0; green, 13; blue, 255 } ,draw opacity=1
] (182.17,182) -- (168.78,191.82) ; \draw [shift={(167.17,193)},
  rotate = 323.75] [color={rgb, 255:red, 0; green, 13; blue, 255 }
  ,draw opacity=1 ][line width=0.75] (6.56,-1.97) .. controls
(4.17,-0.84) and (1.99,-0.18) .. (0,0) .. controls (1.99,0.18) and
(4.17,0.84) .. (6.56,1.97) ;
%Straight Lines [id:da29198388310333045] 
\draw [color={rgb, 255:red, 65; green, 117; blue, 5 } ,draw opacity=1
] (196.08,192.87) -- (189.48,200.49) ; \draw [shift={(188.17,202)},
  rotate = 310.94] [color={rgb, 255:red, 65; green, 117; blue, 5 }
  ,draw opacity=1 ][line width=0.75] (6.56,-1.97) .. controls
(4.17,-0.84) and (1.99,-0.18) .. (0,0) .. controls (1.99,0.18) and
(4.17,0.84) .. (6.56,1.97) ;
%Straight Lines [id:da9217469948945557] 
\draw [color={rgb, 255:red, 4; green, 0; blue, 255 } ,draw opacity=1 ]
(282.86,188.07) -- (292.76,198.08) ; \draw [shift={(294.17,199.5)},
  rotate = 225.31] [color={rgb, 255:red, 4; green, 0; blue, 255 }
  ,draw opacity=1 ][line width=0.75] (6.56,-1.97) .. controls
(4.17,-0.84) and (1.99,-0.18) .. (0,0) .. controls (1.99,0.18) and
(4.17,0.84) .. (6.56,1.97) ;
%Straight Lines [id:da9680792977793742] 
\draw [color={rgb, 255:red, 65; green, 117; blue, 5 } ,draw opacity=1
] (293.42,174.47) -- (301.39,178.58) ; \draw [shift={(303.17,179.5)},
  rotate = 207.31] [color={rgb, 255:red, 65; green, 117; blue, 5 }
  ,draw opacity=1 ][line width=0.75] (6.56,-1.97) .. controls
(4.17,-0.84) and (1.99,-0.18) .. (0,0) .. controls (1.99,0.18) and
(4.17,0.84) .. (6.56,1.97) ;
%Straight Lines [id:da7592534731051833] 
\draw [color={rgb, 255:red, 4; green, 0; blue, 255 } ,draw opacity=1 ]
(171.09,136.96) -- (155.16,136.11) ; \draw [shift={(153.17,136)},
  rotate = 363.07] [color={rgb, 255:red, 4; green, 0; blue, 255 }
  ,draw opacity=1 ][line width=0.75] (6.56,-1.97) .. controls
(4.17,-0.84) and (1.99,-0.18) .. (0,0) .. controls (1.99,0.18) and
(4.17,0.84) .. (6.56,1.97) ;
%Straight Lines [id:da0330093962741953] 
\draw [color={rgb, 255:red, 65; green, 117; blue, 5 } ,draw opacity=1
] (171,149.95) -- (161.17,149.99) ; \draw [shift={(159.17,150)},
  rotate = 359.77] [color={rgb, 255:red, 65; green, 117; blue, 5 }
  ,draw opacity=1 ][line width=0.75] (6.56,-1.97) .. controls
(4.17,-0.84) and (1.99,-0.18) .. (0,0) .. controls (1.99,0.18) and
(4.17,0.84) .. (6.56,1.97) ;
%Straight Lines [id:da014961923755437145] 
\draw [color={rgb, 255:red, 4; green, 0; blue, 255 } ,draw opacity=1 ]
(288.09,105.6) -- (298.44,99.51) ; \draw [shift={(300.17,98.5)},
  rotate = 509.55] [color={rgb, 255:red, 4; green, 0; blue, 255 }
  ,draw opacity=1 ][line width=0.75] (6.56,-1.97) .. controls
(4.17,-0.84) and (1.99,-0.18) .. (0,0) .. controls (1.99,0.18) and
(4.17,0.84) .. (6.56,1.97) ;
%Straight Lines [id:da054090511812718955] 
\draw [color={rgb, 255:red, 65; green, 117; blue, 5 } ,draw opacity=1
] (275.53,92.67) -- (282.71,85.87) ; \draw [shift={(284.17,84.5)},
  rotate = 496.58] [color={rgb, 255:red, 65; green, 117; blue, 5 }
  ,draw opacity=1 ][line width=0.75] (6.56,-1.97) .. controls
(4.17,-0.84) and (1.99,-0.18) .. (0,0) .. controls (1.99,0.18) and
(4.17,0.84) .. (6.56,1.97) ;
%Straight Lines [id:da20709477222668538] 
\draw [color={rgb, 255:red, 4; green, 0; blue, 255 } ,draw opacity=1 ]
(195.17,94) -- (181.49,78.5) ; \draw [shift={(180.17,77)}, rotate =
  408.58000000000004] [color={rgb, 255:red, 4; green, 0; blue, 255 }
  ,draw opacity=1 ][line width=0.75] (6.56,-1.97) .. controls
(4.17,-0.84) and (1.99,-0.18) .. (0,0) .. controls (1.99,0.18) and
(4.17,0.84) .. (6.56,1.97) ;
%Straight Lines [id:da616186312563067] 
\draw [color={rgb, 255:red, 65; green, 117; blue, 5 } ,draw opacity=1
] (183.59,104.62) -- (175.8,99.15) ; \draw [shift={(174.17,98)},
  rotate = 395.12] [color={rgb, 255:red, 65; green, 117; blue, 5 }
  ,draw opacity=1 ][line width=0.75] (6.56,-1.97) .. controls
(4.17,-0.84) and (1.99,-0.18) .. (0,0) .. controls (1.99,0.18) and
(4.17,0.84) .. (6.56,1.97) ;
%Straight Lines [id:da27027429645133494] 
\draw [color={rgb, 255:red, 0; green, 13; blue, 255 } ,draw opacity=1
] (241.17,78) -- (242.07,60) ; \draw [shift={(242.17,58)}, rotate =
  452.86] [color={rgb, 255:red, 0; green, 13; blue, 255 } ,draw
  opacity=1 ][line width=0.75] (6.56,-1.97) .. controls (4.17,-0.84)
and (1.99,-0.18) .. (0,0) .. controls (1.99,0.18) and (4.17,0.84)
.. (6.56,1.97) ;
%Straight Lines [id:da2297200609458736] 
\draw [color={rgb, 255:red, 65; green, 117; blue, 5 } ,draw opacity=1
] (226.17,79) -- (225.39,71.99) ; \draw [shift={(225.17,70)}, rotate =
  443.66] [color={rgb, 255:red, 65; green, 117; blue, 5 } ,draw
  opacity=1 ][line width=0.75] (6.56,-1.97) .. controls (4.17,-0.84)
and (1.99,-0.18) .. (0,0) .. controls (1.99,0.18) and (4.17,0.84)
.. (6.56,1.97) ;
%Straight Lines [id:da33177653749083236] 
\draw [color={rgb, 255:red, 4; green, 0; blue, 255 } ,draw opacity=1 ]
(298.91,147.88) -- (310.17,148.41) ; \draw [shift={(312.17,148.5)},
  rotate = 182.66] [color={rgb, 255:red, 4; green, 0; blue, 255 }
  ,draw opacity=1 ][line width=0.75] (6.56,-1.97) .. controls
(4.17,-0.84) and (1.99,-0.18) .. (0,0) .. controls (1.99,0.18) and
(4.17,0.84) .. (6.56,1.97) ;
%Straight Lines [id:da7070837655966179] 
\draw [color={rgb, 255:red, 65; green, 117; blue, 5 } ,draw opacity=1
] (300.69,134.9) -- (308.19,133.79) ; \draw [shift={(310.17,133.5)},
  rotate = 531.61] [color={rgb, 255:red, 65; green, 117; blue, 5 }
  ,draw opacity=1 ][line width=0.75] (6.56,-1.97) .. controls
(4.17,-0.84) and (1.99,-0.18) .. (0,0) .. controls (1.99,0.18) and
(4.17,0.84) .. (6.56,1.97) ;
%Straight Lines [id:da9465224193534777] 
\draw [color={rgb, 255:red, 4; green, 0; blue, 255 } ,draw opacity=1 ]
(233.17,208) -- (232.63,222.1) ; \draw [shift={(232.55,224.1)}, rotate
  = 272.19] [color={rgb, 255:red, 4; green, 0; blue, 255 } ,draw
  opacity=1 ][line width=0.75] (6.56,-1.97) .. controls (4.17,-0.84)
and (1.99,-0.18) .. (0,0) .. controls (1.99,0.18) and (4.17,0.84)
.. (6.56,1.97) ;
%Straight Lines [id:da700311003901801] 
\draw [color={rgb, 255:red, 65; green, 117; blue, 5 } ,draw opacity=1
] (248.64,204.91) -- (250.72,214.05) ; \draw [shift={(251.17,216)},
  rotate = 257.17] [color={rgb, 255:red, 65; green, 117; blue, 5 }
  ,draw opacity=1 ][line width=0.75] (6.56,-1.97) .. controls
(4.17,-0.84) and (1.99,-0.18) .. (0,0) .. controls (1.99,0.18) and
(4.17,0.84) .. (6.56,1.97) ;
%Straight Lines [id:da29842040596141506] 
\draw [line width=1.5] (114.02,144) -- (356.19,144) ;
%Shape: Polygon Curved [id:ds08608026786206713] 
\draw [fill={rgb, 255:red, 248; green, 231; blue, 28 } ,fill opacity=1
] (226.17,144) .. controls (233.4,145.13) and (236.17,144)
.. (243.17,144) .. controls (250.17,144) and (265.52,156.16)
.. (234.17,156) .. controls (202.82,155.84) and (218.93,142.87)
.. (226.17,144) -- cycle ;


% Text Node
\draw (167,114.4) node [anchor=north west][inner sep=0.75pt] {$\Sigma
  $};
% Text Node
\draw (243.17,149) node [font=\scriptsize] {$\boldsymbol{Q}$};
% Text Node
\draw (220,143.4) node [anchor=north west][inner sep=0.75pt]
      [font=\footnotesize] {$\mathcal{R}$};

\end{tikzpicture}

  \caption{Region $\mathcal{R}$ en la superficie de un conductor
  semi-infinito en la que hay una carga $Q$, la cual produce un campo eléctrico
  $\vec{E}(\vec{r},t)$ (líneas azules) y éste a su vez produce una densidad de
  corriente (líneas verdes) que atraviesan aquella parte de la
  superficie lejana $\Sigma$ que se halla dentro del
  conductor.}
\label{Surfplasmon}
\end{figure}

Usando la ecuación dinámica de la densidad de corriente podemos
escribir la ecuación dinámica de la carga como hicimos en el caso del plasmón de
bulto, con la diferencia que la densidad de corriente en este caso
sólo fluye a través de la mitad de la superficie $\Sigma $ que se
halla en el interior del metal,
\begin{equation}
  \begin{split}
    \frac{d^{2}}{dt^{2}}Q = & -\int_{\Sigma}d\vec{a}\cdot\frac{\partial}{\partial t}\vec{j},\\
    = & ne \int_{\Sigma/2}d\vec{a}\cdot\frac{\partial}{\partial t}\vec{v},\\
    = & \frac{1}{2}\frac{ne^{2}}{m}\int_{\Sigma}d\vec{a}\cdot \vec{E}, \\
    = & -\frac{2\pi ne^{2}}{m}Q.
  \end{split}
\end{equation}
Comparando esta ecuación con la ec. \eqref{OAEq}
identificamos  la frecuencia del {\em plasmón de superficie}
$\omega_{\text{sp}}$, dada por
\begin{equation}
  \omega_{sp}^{2}=\frac{2\pi ne^{2}}{m} = \frac{\omega_{p}^{2}}{2}.
  \label{surfaceplasmonfrecuency}
\end{equation}

En lugar de un sistema semiinfinito, consideremos ahora un sistema
finito consistente en una partícula metálica separada del vacío por
una superficie esférica. Supongamos que perturbamos esta esfera
moviendo todos sus electrones una separación $\vec{\zeta}$ respecto a
su posición de equilibrio, lo cual
induce una polarización $\vec{\mathcal{P}}=-n_0 e\vec \zeta$, como ilustra la
siguiente figura,
\begin{figure}
  \centering
  \tikzset{every picture/.style={line width=0.75pt}} %set default line width to 0.75pt

\begin{tikzpicture}[x=0.75pt,y=0.75pt,yscale=-1,xscale=1]
%uncomment if require: \path (0,300); %set diagram left start at 0, and has height of 300
%Shape: Circle [id:dp14966767726976038]
\draw [fill={rgb, 255:red, 248; green, 231; blue, 28 } ,fill opacity=1
] (219.75,132.08) .. controls (219.75,101.2) and (244.78,76.17)
.. (275.67,76.17) .. controls (306.55,76.17) and (331.58,101.2)
.. (331.58,132.08) .. controls (331.58,162.97) and (306.55,188)
.. (275.67,188) .. controls (244.78,188) and (219.75,162.97)
.. (219.75,132.08) -- cycle ;
%Shape: Circle [id:dp9677285416132982]
\draw [fill={rgb, 255:red, 248; green, 231; blue, 28 } ,fill opacity=1
][dash pattern={on 4.5pt off 4.5pt}] (219.75,108.08) .. controls
(219.75,77.2) and (244.78,52.17) .. (275.67,52.17) .. controls
(306.55,52.17) and (331.58,77.2) .. (331.58,108.08) .. controls
(331.58,138.97) and (306.55,164) .. (275.67,164) .. controls
(244.78,164) and (219.75,138.97) .. (219.75,108.08) -- cycle ;
%Shape: Path Data [id:dp41415328569096044]
\draw [draw opacity=0][fill={rgb, 255:red, 25; green, 23; blue, 2 }
  ,fill opacity=0.53 ] (275.67,164) .. controls (248.27,164) and
(225.42,145.14) .. (220.17,120.08) .. controls (225.42,95.03) and
(248.27,76.17) .. (275.67,76.17) .. controls (303.06,76.17) and
(325.91,95.03) .. (331.17,120.08) .. controls (325.91,145.14) and
(303.06,164) .. (275.67,164) -- cycle ;

%Straight Lines [id:da9936815684711549]
\draw [line width=1.5] (273.03,112.79) -- (273.17,126.96) ; \draw
      [shift={(273,109.79)}, rotate = 89.44] [color={rgb, 255:red, 0;
          green, 0; blue, 0 } ][line width=1.5] (8.53,-2.57)
      .. controls (5.42,-1.09) and (2.58,-0.23) .. (0,0) .. controls
      (2.58,0.23) and (5.42,1.09) .. (8.53,2.57) ;
%Straight Lines [id:da577848764958209]
\draw [color={rgb, 255:red, 0; green, 4; blue, 243 } ,draw opacity=1
][line width=1.5] (235.87,105.83) -- (236.25,134.39) ; \draw
      [shift={(235.83,102.83)}, rotate = 89.24] [color={rgb, 255:red,
          0; green, 4; blue, 243 } ,draw opacity=1 ][line width=1.5]
      (8.53,-2.57) .. controls (5.42,-1.09) and (2.58,-0.23) .. (0,0)
      .. controls (2.58,0.23) and (5.42,1.09) .. (8.53,2.57) ;
%Straight Lines [id:da23709733272432076]
\draw [color={rgb, 255:red, 40; green, 0; blue, 255 } ,draw opacity=1
][line width=1.5] (312.79,105.99) -- (313.17,134.55) ; \draw
      [shift={(312.75,102.99)}, rotate = 89.24] [color={rgb, 255:red,
          40; green, 0; blue, 255 } ,draw opacity=1 ][line width=1.5]
      (8.53,-2.57) .. controls (5.42,-1.09) and (2.58,-0.23) .. (0,0)
      .. controls (2.58,0.23) and (5.42,1.09) .. (8.53,2.57) ;
%Straight Lines [id:da356188359843302]
\draw [color={rgb, 255:red, 22; green, 0; blue, 255 } ,draw opacity=1
][line width=1.5] (256.85,106.76) -- (257,135.32) ; \draw
      [shift={(256.83,103.76)}, rotate = 89.7] [color={rgb, 255:red,
          22; green, 0; blue, 255 } ,draw opacity=1 ][line width=1.5]
      (8.53,-2.57) .. controls (5.42,-1.09) and (2.58,-0.23) .. (0,0)
      .. controls (2.58,0.23) and (5.42,1.09) .. (8.53,2.57) ;
%Straight Lines [id:da517585674881536]
\draw [color={rgb, 255:red, 22; green, 0; blue, 255 } ,draw opacity=1
][line width=1.5] (292.02,106.91) -- (292.17,135.47) ; \draw
      [shift={(292,103.91)}, rotate = 89.7] [color={rgb, 255:red, 22;
          green, 0; blue, 255 } ,draw opacity=1 ][line width=1.5]
      (8.53,-2.57) .. controls (5.42,-1.09) and (2.58,-0.23) .. (0,0)
      .. controls (2.58,0.23) and (5.42,1.09) .. (8.53,2.57) ;
%Straight Lines [id:da6814351373246353]
\draw [color={rgb, 255:red, 126; green, 211; blue, 33 } ,draw
  opacity=1 ][line width=0.75] (283.14,134.98) -- (282.7,105.43) ;
\draw [shift={(283.17,136.98)}, rotate = 269.15] [color={rgb, 255:red,
    126; green, 211; blue, 33 } ,draw opacity=1 ][line width=0.75]
(8.74,-2.63) .. controls (5.56,-1.12) and (2.65,-0.24) .. (0,0)
.. controls (2.65,0.24) and (5.56,1.12) .. (8.74,2.63) ;
%Straight Lines [id:da19196139788350697]
\draw [color={rgb, 255:red, 126; green, 211; blue, 33 } ,draw
  opacity=1 ][line width=0.75] (250.16,135.21) -- (249.72,105.65) ;
\draw [shift={(250.19,137.21)}, rotate = 269.15] [color={rgb, 255:red,
    126; green, 211; blue, 33 } ,draw opacity=1 ][line width=0.75]
(8.74,-2.63) .. controls (5.56,-1.12) and (2.65,-0.24) .. (0,0)
.. controls (2.65,0.24) and (5.56,1.12) .. (8.74,2.63) ;
%Straight Lines [id:da7557269004865166]
\draw [color={rgb, 255:red, 126; green, 211; blue, 33 } ,draw
  opacity=1 ][line width=0.75] (299.33,134.18) -- (299.06,104.62) ;
\draw [shift={(299.35,136.18)}, rotate = 269.48] [color={rgb, 255:red,
    126; green, 211; blue, 33 } ,draw opacity=1 ][line width=0.75]
(8.74,-2.63) .. controls (5.56,-1.12) and (2.65,-0.24) .. (0,0)
.. controls (2.65,0.24) and (5.56,1.12) .. (8.74,2.63) ;
%Straight Lines [id:da6683208735835106]
\draw [color={rgb, 255:red, 126; green, 211; blue, 33 } ,draw
  opacity=1 ][line width=0.75] (266.08,135.02) -- (265.81,105.47) ;
\draw [shift={(266.1,137.02)}, rotate = 269.48] [color={rgb, 255:red,
    126; green, 211; blue, 33 } ,draw opacity=1 ][line width=0.75]
(8.74,-2.63) .. controls (5.56,-1.12) and (2.65,-0.24) .. (0,0)
.. controls (2.65,0.24) and (5.56,1.12) .. (8.74,2.63) ;

% Text Node
\draw (268,47.4) node [anchor=north west][inner sep=0.75pt] {$-$};
% Text Node
\draw (285,52.4) node [anchor=north west][inner sep=0.75pt] {$-$};
% Text Node
\draw (298,58.4) node [anchor=north west][inner sep=0.75pt] {$-$};
% Text Node
\draw (307,66.4) node [anchor=north west][inner sep=0.75pt] {$-$};
% Text Node
\draw (314,79.4) node [anchor=north west][inner sep=0.75pt] {$-$};
% Text Node
\draw (232,66.4) node [anchor=north west][inner sep=0.75pt] {$-$};
% Text Node
\draw (226,76.4) node [anchor=north west][inner sep=0.75pt] {$-$};
% Text Node
\draw (240,57.4) node [anchor=north west][inner sep=0.75pt] {$-$};
% Text Node
\draw (252,51.4) node [anchor=north west][inner sep=0.75pt] {$-$};
% Text Node
\draw (222,139.4) node [anchor=north west][inner sep=0.75pt] {$+$};
% Text Node
\draw (231,153.4) node [anchor=north west][inner sep=0.75pt] {$+$};
% Text Node
\draw (241,163.4) node [anchor=north west][inner sep=0.75pt] {$+$};
% Text Node
\draw (253,168.4) node [anchor=north west][inner sep=0.75pt] {$+$};
% Text Node
\draw (284.32,167.86) node [anchor=north west][inner sep=0.75pt]
      [rotate=-359.43] {$+$};
% Text Node
\draw (298.37,159.96) node [anchor=north west][inner sep=0.75pt]
      [rotate=-359.37] {$+$};
% Text Node
\draw (309.32,148.14) node [anchor=north west][inner sep=0.75pt]
      [rotate=-358.41] {$+$};
% Text Node
\draw (316.41,137.18) node [anchor=north west][inner sep=0.75pt]
      [rotate=-358.48] {$+$};
% Text Node
\draw (268,169.4) node [anchor=north west][inner sep=0.75pt] {$+$};
% Text Node
\draw (273,119.4) node [anchor=north west][inner sep=0.75pt]
      [font=\small] [color=white] {$\vec{\mathbf{\zeta }}$};

\end{tikzpicture}


    \caption{Esfera metálica cuyos electrones han sido desplazados en la
  dirección de $\vec{\zeta}$ (flecha negra), dejando una ausencia de
  carga en la dirección opuesta induciendo polarización en el sistema
  $\vec{\mathcal{P}}$ (flechas verdes) y una carga superficial la cual
  produce un campo eléctrico $\vec{E}(\vec{r},t)$.}
\label{Polariton}

\end{figure}
El deplazamiento de los electrones hacia uno de los
hemisferios de la esfera genera un exceso de carga negativa en su
superficie y un exceso de carga positiva en el otro, descrita por la
densidad de carga superficial $\sigma=\vec{\mathcal P}\cdot\hat n$,
donde $\hat n$ es un vector unitario radial. Estas cargas producen un
campo eléctrico
$\vec{E}(\vec{r},t) = -(4\pi/3)\vec{\mathcal{P}}
$ donde empleamos el {\em factor de depolarización} $-4\pi/3$ de una
  esfera. Este campo acelera las cargas de acuerdo a
$m\frac{d^{2}\vec{\zeta}}{d^{2}t}=-e\vec{E}(\vec{r},t)$. Escribiendo
al campo en términos de $\vec\zeta$, $\vec{E}= -(4\pi n_0
  e^{2}/3m)\vec{\zeta}$, la ecuación de movimiento para
$\vec{\zeta}$ se convierte en,
\begin{equation}
  \frac{d^{2}\vec{\zeta}}{d^{2}t}= -\frac{4\pi_0 e^{2}}{3m}\vec{\zeta}
  \label{polaritondifec}
\end{equation}
de la cual obtenemos la frecuencia de las oscilaciones de carga en una
esfera, a las que se les denominan como {\em plasmón dipolar},
\begin{equation}
  \omega_{d}^{2} = \frac{4\pi n e^{2}}{3m} = \frac{\omega^2_{p}}{3}.
\end{equation}

Estos ejemplos muestran que en un metal los electrones pueden animarse
de movimientos colectivos asociados a ciertas frecuencias de
resonancia, las cuales a su vez dependen de la geometría, como
ilustramos estudiando
el caso de un sistema infinito, uno semiinfinito y una esfera. Más
aún, si colocamos las partículas metálicas en el seno de una matriz
dieléctrica habría un corrimiento adicional en su frecuencia de
resonancia debido a las cargas inducidas en la superficie del
dieléctrico. Más aún, si hubiese un gran número de esferas, sus
interacciones mutuas a través de los campos electromagnéticos
inducidos podrían generar corrimientos adicionales de las resonancias.

Un ejemplo excepcional de las propiedades que emergen al generar
\textit{metamateriales} es la copa de Lycurgo, una copa de cristal
tallada en la época romana tardía, decorada con un friso que muestra
escenas del mito del Rey Lycurgo. La copa que data del siglo IV, D.C.,
se produjo a partir de una pieza en bruto de vidrio soplado de unos
15mm de espesor. Las figuras se cortaron y rectificaron y se unieron a
la pared del recipiente mediante pequeños puentes de vidrio. Aparte
del trabajo artístico realizado en la decoración, la copa es de gran
interés por las propiedades ópticas que muestra. El vidrio aparece de
un rojo-vino profundo cuando la luz lo atraviesa y de un verde opaco
cuando la luz que llega a nuestros ojos es reflejada por la superficie
de la copa, como muestra la fig. \ref{Lycurgus}. A este fenómeno se
le denomina \textit{dicroísmo}, y de los artefactos de vidrio romano
encontrados, la copa es la que muestra dicho efecto más
intensamente. \cite{LycurgusInvestigation}
\begin{figure}
    \centering
    \includegraphics[width = 0.6\textwidth]{Lycurguscup.jpg}
    \caption{Copa de Lycurgo, fabricada en la Roma tardía del
      siglo IV D.C., cuya fabricación resultante muestra propiedades
      ópticas excepcionales.  Es una taza para beber de vidrio que se
      ve verde o rojo, dependiendo de cómo es iluminada.
      ©The Trustees of the British Museum.}
    \label{Lycurgus}
\end{figure}
Estudios de la composición del vidrio muestran que
tales propiedades son causadas por la presencia de finas partículas de
oro, probablemente en una aleación con plata, dispersadas. Con estudios de
microscopía de transmisión de electrones, TEM, por sus siglas en
inglés, se pudieron determinar tamaños de las partículas de $\approx
10 nm $.  Se ha encontrado que contiene admás partículas de
diferentes metales y de materiales no metálicos. El color se debe al
espectro de reflexión y de transmisión del medio compuesto formado por
vidrio y por las partículas metálicas. Aunque el oro es amarillo, las
partículas de oro embebidas en una matriz de vidrio e interaccionando
entre sí producen un color rojo.

\subsection{Otras geometrías}
Si consideráramos partículas con otras geometrías habría
otras resonancias asociadas a la excitación de modos con patrones
varios de distribución de carga. Por ejemplo, en la figura
\ref{fig:Fuchs} mostramos resultados experimentales y los primeros
resultados teóricos para los modos electromagnéticos esperados en
pequeños cubos de sal \cite{Fuchs}, sus frecuencias de resonancia y su
distribución asociada de carga superficial.
\begin{figure}
  \centering
  \includegraphics[width=0.4\textwidth]{fuchs1}
  \includegraphics[width=0.5\textwidth]{fuchs2}
  \caption{Modos resonantes de un cristal de sal y espectro de
    absorción teórico y experimental. Del lado izquierdo mostramos los
  signos de la polarización en una de las ocho esquinas del cubo de
  sal correspondiente a los seis modos principales.}
  \label{fig:Fuchs}
\end{figure}
En este caso se encontraron en lugar de un modo dipolar, como vimos
para el caso de la esfera, seis modos principales y unos modos
adicionales con poca fuerza de oscilador, con una polarización cuya
distribución espacial muestra bastante riqueza.
\subsection{Cristales fotónicos}
Consideremos ahora un dieléctrico transparente no dispersivo
homogéneo, como en la figura \ref{fig:fotonico}a.
\begin{figure}
  \centering
  \includegraphics[width=0.3\textwidth]{talk-9}
  \includegraphics[width=0.3\textwidth]{talk-10}
  \includegraphics[width=0.3\textwidth]{talk-11}
  \\
  \includegraphics[width=0.3\textwidth]{wvsk1-0}
  \includegraphics[width=0.3\textwidth]{wvsk1-1}
  \includegraphics[width=0.3\textwidth]{wvsk1-2}\\
  \begin{picture}(0,0)
    \put(-6.5cm,8.1cm){\color{white} (a)}
    \put(-2.0cm,8.1cm){\color{white} (b)}
    \put(2.5cm,8.1cm){(c)}
    \put(-6.0cm,3.5cm){(d)}
    \put(-1.5cm,3.5cm){(e)}
    \put(3cm,3.5cm){(f)}
  \end{picture}
  \caption{(a)Medio dieléctrico homogéneo con respuesta $\epsilon$,
    (b) cristal fotónico unidimensional con funciones respuesta
    $\epsilon_a$ y $\epsilon_b$ y periodo $d$, y (c) cristal fotónico
    con periodicidad en más dimensiones. (d)Relación de
    dispersión $\omega$ vs $k$ de la luz del medio
    homogéneo. (e)Relaciones de dispersión transladadas por $0, \pm
    2\pi/d$ en el espacio recíproco mostrando cruzamientos en el borde
    de la zona de Brillouin $k=\pm \pi/d$. (f)El acoplamiento abre una
    brecha en los puntos de degeneración evitando el cruce y formando
    brechas prohibidas.}
  \label{fig:fotonico}
\end{figure}
La relación de dispersión de la luz en este medio está dada por $k^2=
\epsilon\omega^2/c^2$ que corresponde a las dos rectas mostradas en la
fig. \ref{fig:fotonico}d. Si en vez de un dieléctrico homogéneo
tuvieramos un cristal artificial formado por películas de dos
materiales alternados
(fig. \ref{fig:fotonico}b), el ímpetu y el vector de onda ya no
serían cantidades conservadas. Las
reflecciones múltiples en las interfaces producirían
ondas esparcidas en que el vector de onda cambiaría $k\to k+2\pi n/d$
para enteros positivos y negativos $n$
(fig. \ref{fig:fotonico}e), dando lugares a puntos de degeneración en
que se cruzan las distintas réplicas de la relación de dispersión. El
acoplamiento entre los campos esparcidos rompe la degeneración y abre
{\em brechas fotónicas}  evitando los cruces y dando origen a una
relación de dispersión (fig. \ref{fig:fotonico}f) organizada en {\em
  bandas fotónicas} análogas a las bandas electrónicas que describen
la propagación de electrones en sólidos cristalinos. Algo similar
suscedería si la periodicidad fuese bidimensional o
tridimensional (fig. \ref{fig:fotonico}c) en cuyo caso podría
producirse brechas omnidireccionales en las que la luz no se propaga
en ninguna dirección. Las brechas fotónicas explican algunos fenómenos
naturales, como la iridiscencia en los caparazones de diversos
insectos y los colores de las alas de las mariposas, colores
producidos no por pigmentos que absorben la luz, sino por pequeñas estructuras
dieléctricas transparentes que forman cristales fotónicos con regiones
de frecuencia en que la luz es fuertemente reflejada por corresponder
a brechas en que no se puede propagar. Estos colores se llaman por su
origen {\em colores estructurales}. Introduciendo {\em defectos} en cristales
fotónicos se pueden generarse sitios en que la luz puede ser atrapada,
hecho que ha encontrado aplicaciones tal
y como la elaboración de fibras ópticas fotónicas.

\subsection{Materiales izquierdos}
Consideremos ahora un material cuya permitividad $\epsilon(\omega)$
dependa de la frecuencia y tenga un comportamiento resonante. La
relación de dispersión $k^2=\epsilon\omega^2/c^2$ implica que al pasar
la resonancia, cuando $\epsilon$ adquiere valores negativos, $k$ se
vuelve imaginario y la luz no puede propgarse. Esto explica la
aparición del color en los materiales comunes, en que hay frecuencias
caracterizticas de cada material en absorben la luz y justo arriba hay
frecuencias en que no la pueden propagar. Si el material tuviera
además una respuesta magnética $\mu\ne 1$ la relación de dispersión
cambiaría a $k^2=\epsilon\mu\omega^2/c^2$. Si tanto
$\epsilon$ como $\mu$ tuvieran resonancias cercanas, arriba de éstas
podría suceder que ambas fueran negativas, $\epsilon<0$ y $\mu<0$. En
este caso, su producto $\epsilon\mu>0$ sería positivo y sí podría
haber propagación con un vector de onda real. Sin embargo, esta
propagación sería curiosa. Sabemos que a partir de las
ecuaciones de Maxwell, por ejemplo, de las leyes de Faraday y de Gauss magnética,
que para una onda plana, el campo eléctrico $\vec E$, la densidad de
flujo magnético
$\vec B$ y el vector de onda $\vec k$ forman una triada ordenada
derecha, como ilustra la figura \ref{fig:triada}.
\begin{figure}
  \centering
  \includegraphics{talk-13}
  \caption{Relación entre las direcciones del campo eléctrico
    $\vec E$, densidad de flujo magnético $\vec B$, vector de onda
    $\vec k$, campo magnético $\vec H$ y vector de Poynting $\vec S$
    en un metamaterial izquierdo isotrópico. Como $(\vec E, \vec B,
    \vec k)$ y $(\vec E, \vec H,\vec S)$ son triadas ordenadas
    derechas, y $\vec B$ y $\vec H$ son antiparalelos, entonces $\vec
    k$ y $\vec S$ son antiparalelos.  }
  \label{fig:triada}
\end{figure}
Sin embargo, la definición del vector de Poynting y la ley de
Ampere-Maxwell implican que el campo eléctrico $\vec E$, el campo
magnético
$\vec H$ y el flujo de energía $\vec S$ también forman una triada
ordenada derecha. Sin embargo, si $\mu<0$, entonces $\vec B$ y $\vec
H$ apuntan en direcciones opuestas. Luego, $\vec S$ ¡apunta en la
dirección opuesta a $\vec k$! La dirección en que avanza la fase de la
onda es opuesta a la dirección en que avanza la energía. Esto sólo
puede ser posible si la velocidad de grupo es opuesta a la velocidad
de fase. Una consecuencia curiosa de este resultado se manifiesta
cuando una onda se refracta en una superficie plana. La ley de
conservación del ímpetu asociada a una simetría translacional implica
que la proyección del vector de onda $\vec k_\|$ a lo largo de la
superficie debe coicidir para la onda reflejada, la onda transmitida y
la onda incidente. De aquí se derivan las leyes de la reflexión y de
Snell. Sin embargo, la {\em causalidad} implica que las ondas
esparcidas por la superficie, la onda incidente y la onda reflejada,
deben tener un flujo de energía que se aleja de la superficie. Ello
implica que cuando incide luz desde un medio ordinario hacia un medio
con $\epsilon<0$ y $\mu<0$, la componente normal del vector de onda
de la onda transmitida ¡debe apuntar hacia la superficie!, como
ilustrar la fig. \ref{fig:refneg}.
\begin{figure}
  \centering
  \includegraphics{talk-17}
  \caption{Una onda incide desde un medio normal (verde) sobre la
    interface que lo separa de un medio con permitividad y
    permeabilidad negativas (rojo). Se muestran esquemáticamente las
    direcciones de los vectores de onda (flechas azules) y de los
    vectores de Poynting (flechas amarillas) de las ondas incidente,
    reflejada y transmitida. La proyección sobre la interface de los
    tres vectores de onda debe coincidir. El flujo de energía de la
    onda transmitida debe alejarse de la interface. Por ello, la
    componente normal del vector de onda transmitido apunta hacia la
    interface.}
  \label{fig:refneg}
\end{figure}
De esta figura podemos inferir que una onda que incide viajando hacia
arriba se refracta hacia abajo y vieversa. Eso lleva a plantear
dispositivos como el ilustrado en la fig. \ref{fig:lenteplana}
consistente en una película plana de un metamaterial izquierdo con
$\epsilon<0$ y $\mu<0$.
\begin{figure}
  \centering
  \includegraphics{talk-24}
  \caption{Lente consistente en una película plana de un metamaterial
    izquierdo (rojo) en el seno de un material ordinario (verde). Se
    ilustra la trayectoria de varios rayos que emergen de una fuente puntual de
    luz y convergen en un punto imagen. }
  \label{fig:lenteplana}
\end{figure}
La luz que emerge de una fuente puntual y viaja hacia la derecha y
hacia arriba se refracta hacia abajo mientras que luz que viaja hacia
abajo se refracta hacia arriba. Algo análogo sucede al emerger de la
película. Es posible entonces que todos los rayos que parten de la
fuente luminosa converjan en un punto, la imagen de la fuente formada
por una lente plana.

Desafortunadamente, no existen materiales naturales en los que tanto
la permitividad como la permeabilidad sean negativas a la misma
frecuencia. Sin embargo, hay metamateriales artificiales que pueden
describirse por una permitividad y permeabilidad efectiva que sí
cumplan esta condición. La figura \ref{fig:rings} muestra un ejemplo
\begin{figure}
  \centering
  \includegraphics[width=.7\textwidth]{splitResonator}
  \caption{Metamaterial izquierdo formado por una red de parejas de
    anillos conductores interrumpidos ({\em split rings}) y pistas
    conductoras rectas sobre un dieléctrico. }
  \label{fig:rings}
\end{figure}
formado por un arreglo de parejas de anillos interrumpidos que
funcionan como un circuito LC resonante. La corriente recorriendo los anillos
produce un dipolo magnético, y debido a su interrupción, produce una
acumulación de cargas que lo acopla con el otro anillo. Este circuito
tiene una resonancia arriba de la cual la permeabilidad macroscópica
es negativa. Por otro lado, una serie de pistas rectas permiten que el
material se comporte en la dirección vertical como un conductor, por
lo cual la permitividad es negativa abajo de la frecuencia de plasma
efectiva.
\subsection{Metasuperficies}
Hemos visto arriba que partículas metálicas pequeñas pueden tener resonancias
plasmónicas cuyas frecuencias dependen en general de su composición y
de su geometría. También partículas dieléctricas pueden tener
resonancias aunque estén formadas por materiales no dispersivos,
siempre y cuando la longitud de onda de la luz en su interior sea
conmensurable con su tamaño. Estas resonancias se deben a la
interferencia constructiva entre ondas múltiplemente reflejadas por
sus superficies. Por ejemplo, partículas esféricas o cilíndricas
muestran {\em resonancias de Mie} cuando el perímetro de su sección
transversal es cercano a un múltipolo de la longitud de onda.
Una ventaja de estas resonancias  sobre las
resonancias plasmónicas para diseñar y
construir dispositivos fotónicos  es
que las pérdidas de energía debidas a la absorción dentro del material
son menores que las pérdidas ohmicas que suelen mostrar los
metales. Sin embargo, estas resonancias requieren que las partículas
tengan un tamaño relativamente grande conmensurable con la longitud de
onda en su interior. Sin embargo, si se emplean
materiales con un índice de refracción alto, la longitud de onda
dentro de estos materiales puede ser mucho menor que la
correspondiente al espacio vacío, permitiéndo así resonancias
dieléctricas en partículas de tamaño muy pequeñas, en analogía a las
resonancias plasmónicas.

Las funciones respuesta de una partícula cambian de
signo conforme la frecuencia de la luz pasa de ser menor a ser mayor a
su frecuencia de resonancia. Por tanto, la fase que adquiere un haz
luminoso al pasar a través de una superficie cubierta por partículas
depende muy sensiblemente de la cercanía de la frecuencia a la
frecuencia de resonancia de las partículas, la cual a su vez, depende de
la geometría. Por tanto, modulando la geometría de las partículas a lo
largo de la superficie, puede modularse la fase que adquiere la luz en
forma análoga a como el ancho variable de una lente o de un prisma
modula la fase de los rayos de luz que los atraviesan.

La ley de Snell usual implica que a lo largo de una interface uniforme
hay un empatamiento de fases $\phi^\alpha(\vec r_\|)=\vec k^\alpha_\|\cdot\vec
r_\|$ entre la onda incidente, la onda reflejada y la
onda transmitida, por lo cual los vectores de onda $\vec k^\alpha_\|$
proyectados sobre la superficie son iguales para las tres ondas $\alpha=i,r,t$. Sin
embargo, si la superficie no es uniforme y a lo largo de ésta la onda
transmitida y/o reflejada adquiere una fase adicional $\psi^\alpha(\vec
r_\|)$ a la de la onda incidente, $\phi^\alpha(\vec
r_\|)=\vec k^i_\|\cdot\vec r_\|+\psi^\alpha(\vec r_\|)\approx (\vec
k^i_\|+\nabla_\|\psi^\alpha(0))\cdot\vec r_\|$ la ley de Snell debe
generalizarse,
\begin{equation}
  \label{eq:snell}
  \vec k^\alpha_\|=\vec k^i_\|+\nabla_\|\psi^\alpha,\quad(\alpha=r,t)
\end{equation}
i.e., el ímpetu paralelo a la interfase adquiere una contribución
debida a la variación de la fase adicional $\psi^\alpha$. Por lo
tanto, modulando la fase de una onda a lo largo de una superficie
podemos manipular la dirección de la luz transmitida o reflejada. Para
esto se pueden colocar partículas con un índice de refracción grande
sobre una superficie ordinaria y modificar a lo largo de esta su
geometría, orientación o densidad, dando lugar a una {\em metasuperficie}.


Como las
resonancias dependen también de la polarización de la luz, este efecto
puede usarse para desviar haces de luz de acuerdo a su
polarización. En la fig. \ref{fig:metasup} mostramos una
{\em metasuperficie} formada por partículas en forma de $L$ cuya
respuesta difiere cuando es iluminada con polarización circular
derecha o polarización circular izquierda, y que por tanto puede
separa un haz de luz no polarizada en dos haces con polarizaciones opuestas.
\begin{figure}
  \centering
  \includegraphics[width=0.5\textwidth,angle=90,valign=c]{metasup1}
  \includegraphics[width=0.4\textwidth,valign=c]{metasup2a}
  \caption{Metasuperficie formada por un arreglo de partículas en
    forma de L con ángulos variables (izquierda), capaz de separar un
    haz luminoso en sus componentes con polarización circular derecha
    e izquierda (derecha).}
  \label{fig:metasup}
\end{figure}
Mediante otros arreglos se puede variar la fase a lo largo de la
dirección radial, de forma de hacer converger rayos que arriben en la
dirección normal sobre un punto, su foco, creando así una metalente, como la
que ilustra la figura \ref{fig:metalente}.
\begin{figure}
  \centering
  \includegraphics[width=.5\textwidth,angle=90, valign=c]{metasup3}
  \includegraphics[height=.3\textwidth,valign=c]{metasup5}
  \includegraphics[height=.5\textwidth,valign=c]{metasup4}
\caption{Metalente formada por un arreglo de prismas de alto índice
    de refracción con direcciones variables (izquierda). Imagen de una
    fuente puntual formada por la metalente y por una lente
    convencional (derecha). }
  \label{fig:metalente}
\end{figure}

\ifignora
\subsection{Electrodinámica}
La descripción de los fenomenos electromagnéticos es provista por las
ecuaciones de Maxwell las cuales describen la propagación de los
campos electromagnéticos en el vacío y en la materia, su interacción
con las cargas eléctricas de los materiales, y la interacción entre las
misma cargas, las ecuaciones en forma diferencial son,
\begin{equation}
  \label{Maxwell-eqs}
  \begin{array}{cc}
    \nabla \cdot {\bf D} = 4\pi \rho_{ex}, & \nabla \cdot {\bf B} =0, \\
    \nabla \times {\bf E} = -\frac{1}{c}\frac{\partial {\bf B}}{\partial t}, &
    \nabla \times {\bf H} = \frac{4\pi}{c}{\bf
      J}_{ex} + \frac{1}{c}\frac{\partial {\bf D}}{\partial t},
  \end{array}
\end{equation}
donde ${\bf J}_{ex}$ y $\rho_{ex}$ son la densidad de corriente y
densidad de carga externas, respectivamente. Por ejemplo, el siguiente
sistema que consiste de circuitos de alambre a los que se aplica un
campo magnético a lo largo de la dirección perpendicular al plano en
el que está el circuito y un campo eléctrico que reside en el mismo
plano y cuyo efecto sobre los circuitos es el movimiento de sus
caragas eléctricas, como ilustra la figura \ref{Circuitos}.
\begin{figure}
  \centering
  \input{circuitos.code.tex}

  \caption{Circuitos de lazos de metal bajo la aplicación de un campo
    magnético $\vec{B}$ en la dirección perpendicular a al plano del
    circuito y de un campo eléctrico $\vec{E}$ que reside en el plano
    del circuito. El efecto de la aplicación de los campos sobre los
    alambres de metal es el movimiento de cargas eléctricas, la
    densidad de corriente $\vec{j}$. }
\label{Circuitos}

\end{figure}

Los efectos en la materia por perturbaciones externas es descrito por
sus propiedades intrínsecas tales como la permitividad
$\epsilon(\vec{k},\omega)$, la cual describe cuánto es afectado un
material al aplicar un campo eléctrico y la permeabilidad
$\mu(\vec{k},\omega)$, que describe el efecto cuando se aplican campos
magnéticos. A estas cantidades se les denomina funciones respuesta del
sistema y pueden o no depender de $\vec{k}$ el vector de onda la cual
es y $\omega$ la frecuencia de la preturbación, en el espació real
estas dependencias son espacial y temporal respectivamente. Las
diferentes formas de estás funciones dan pie a la clasifición de la
materia, por ejemplo, un medio con permitividad constante $\epsilon$
en todas las direcciones es un medio isotrópico homogéneo, los medios
cuyas funciones respuesta dependen de la dirección y/o del punto en
que se realiza la medición, se les clasifica como anisotrópicos
inhomogenéos.

La propagación de ondas electromagnéticas en los medios es descrita
por estas funciones respuesta, por ejemplo, consideremos un medio
homogéneo con permitividad constante $\epsilon$, la propagación de la
onda está dada por la relación entre el vector de onda y la
frecuencia, la relación de dispersión de la onda al propagarse en el
medio como se ilustra en la imagen \ref{mediohomogneo}, la cual es
amplificada por el factor $\sqrt{\epsilon}$ con respecto a la relación
de dispersión en el vacío.
\begin{figure}
  \centering
  

\tikzset{every picture/.style={line width=0.75pt}} %set default line width to 0.75pt        

\begin{tikzpicture}[x=0.75pt,y=0.75pt,yscale=-1,xscale=1]
%uncomment if require: \path (0,300); %set diagram left start at 0, and has height of 300

%Shape: Rectangle [id:dp2370189011099142] 
\draw   (346.17,40) -- (596.17,40) -- (596.17,224.17) -- (346.17,224.17) -- cycle ;
%Straight Lines [id:da9477212198255883] 
\draw [color={rgb, 255:red, 31; green, 166; blue, 236 }  ,draw opacity=1 ]   (346.17,73.17) -- (471.17,224.17) ;
%Straight Lines [id:da38120967955829277] 
\draw [color={rgb, 255:red, 31; green, 166; blue, 236 }  ,draw opacity=1 ]   (596.17,74.17) -- (471.17,224.17) ;

%Straight Lines [id:da8438048623151154] 
\draw [color={rgb, 255:red, 208; green, 2; blue, 27 }  ,draw opacity=1 ]   (346.17,41.17) -- (471.17,223.17) ;
%Straight Lines [id:da6201464061225517] 
\draw [color={rgb, 255:red, 208; green, 2; blue, 27 }  ,draw opacity=1 ]   (596.17,42.37) -- (471.17,223.17) ;

%Straight Lines [id:da8440907794928258] 
\draw    (471.17,223.17) -- (470.17,40.17) ;

%Shape: Rectangle [id:dp06864759116466967] 
\draw  [color={rgb, 255:red, 31; green, 166; blue, 236 }  ,draw opacity=1 ][fill={rgb, 255:red, 31; green, 166; blue, 236 }  ,fill opacity=1 ] (53.17,41) -- (319.17,41) -- (319.17,225.17) -- (53.17,225.17) -- cycle ;



% Text Node
\draw (299,-107.6) node [anchor=north west][inner sep=0.75pt]  [font=\tiny]  {$\vec{\mathbf{\zeta }}$};
% Text Node
\draw (55.17,228.57) node [anchor=north west][inner sep=0.75pt]    {$( a)$};
% Text Node
\draw (84,68.4) node [anchor=north west][inner sep=0.75pt]  [font=\Large]  {$\epsilon $};
% Text Node
\draw (348.17,227.57) node [anchor=north west][inner sep=0.75pt]    {$( b)$};
% Text Node
\draw (519,157.4) node [anchor=north west][inner sep=0.75pt]  [color={rgb, 255:red, 47; green, 43; blue, 198 }  ,opacity=1 ]  {$k=\sqrt{\epsilon }\frac{\omega }{c}$};
% Text Node
\draw (507,71.4) node [anchor=north west][inner sep=0.75pt]  [color={rgb, 255:red, 208; green, 2; blue, 27 }  ,opacity=1 ]  {$k=\frac{\omega }{c}$};
% Text Node
\draw (465,233.4) node [anchor=north west][inner sep=0.75pt]  [color={rgb, 255:red, 0; green, 0; blue, 0 }  ,opacity=1 ]  {$k$};
% Text Node
\draw (327,121.4) node [anchor=north west][inner sep=0.75pt]    {$\omega $};


\end{tikzpicture}


  \caption{(a) Medio homogéneo con permitividad constante $\epsilon$,
    (b) relaciones de dipersión de la propagación de la onda, línea
    roja propagación en el vacío, línea azul propagación dentro del material.}
\label{mediohomogneo}

\end{figure}

\fi %ifignora

\section{Teoría}

Las propiedades ópticas de estos materiales compuestos están determinadas por su
composición y geometría. Propiedades como las relaciones de dispersión
de los modos electromagnéticos que se propagan a través del material
en un sistema semi-infinito, las amplitudes de reflexión y transmisión,
y relaciones de dispersión de modos electromagnéticos en la superficie
de un sistema bordeado pueden ser expresadas en términos del operador
dieléctrico macroscópico a través de las soluciones de las ecuaciones
de Maxwell en dicho material.

Las ecuaciones de Maxwell con las cuales se asume una descripción
macroscópica de los campos en un medio son
\begin{equation}
  \label{Maxwell-eqs1}
  \begin{array}{cc}
    \nabla \cdot {\bf D} = 4\pi \rho_{ex}, & \nabla \cdot {\bf B} =0, \\
    \nabla \times {\bf E} = -\frac{1}{c}\frac{\partial {\bf B}}{\partial t}, &
    \nabla \times {\bf H} = \frac{4\pi}{c}{\bf
      J}_{ex} + \frac{1}{c}\frac{\partial {\bf D}}{\partial t},
  \end{array}
\end{equation}
donde ${\bf J}_{ex}$ y $\rho_{ex}$ son la densidad de corriente y
densidad de carga externas, respectivamente. Y para calcular
respuestas ópticas de un sistema se define al operador dieléctrico
$\hat{\epsilon}_{M}$ a través de la relación
\begin{equation}
  {\bf D}_{a} = \hat{\epsilon}_{M} {\bf E}_{a}
  \label{Def-Dilectric-Op}
\end{equation}
donde ${\bf D}_{a}$ y ${\bf E}_{a}$ representan los promedios macroscópicos del
campo desplazamiento y eléctrico, respectivamente.

Sin embargo es posible definir ${\bf D}$ y ${\bf H}$ para incluir
fluctuaciones microscópicas y que sigan obedeciendo las ecuaciones de
Maxwell \eqref{Maxwell-eqs1}.

Para poder describir la respuesta óptica de un sistema que incluye
fluctuaciones locales de los campos inducidas por inhomogeneidades del
sistema, consideramos primero una respuesta dieléctrica,
$\hat{\epsilon}$, microscópica
\begin{equation}
  {\bf D} = \hat{\epsilon} {\bf E},
  \label{MicroscopicDielectric-Op}
\end{equation}
la cual acopla las fluctuaciones locales de los campos ${\bf D}$ y
${\bf E}$ totales y a la cual se le aplica un procedimiento de
promediación para relacionar $\hat{\epsilon}$ con
$\hat{\epsilon}_{M}$, de tal manera que las fluctuaciones
microscópicas sean incorporadas en la respuesta macroscópica. Para tal
proposito seguimos el procedimiento descrito en
\cite{ElectromagneticResponseofSystemwithSpatialFluctuations}, por
Mochán et al.

Formalmente se definen los operadores
\begin{equation}
  \begin{array}{cc}
    \label{operadoresPromedio-Fluctuaciones}
    \hat{P}_{a}, & \hat{P}_{f} = \hat{1} - \hat{P}_{a},
  \end{array}
\end{equation}
los cuales extraen la componente promedio $F_{a}=\hat{P}_{a}F$ y
fluctuante $F_{f}=\hat{P}_{f}F$ de una función $F = F_{a}+F_{f}$
arbitraria. Tales operadores son idempotentes
\begin{equation}
  \begin{array}{cc}
    \label{Op-Idempotentes}
    \hat{P}^{2}_{a}=\hat{P}_{a}, & \hat{P}^{2}_{f} =\hat{P}_{f},
  \end{array}
\end{equation}
\begin{equation}
  \hat{P}_{a}\hat{P}_{f}=\hat{P}_{f}\hat{P}_{a}=0,
\end{equation}
lo cual significa que son operadores de proyección.

Consideremos un campo eléctrico consistente de una parte promedio y
una parte fluctuante
\begin{equation}
  {\bf E} =
  \begin{pmatrix}
  {\bf E}_{a} \\
  {\bf E}_{f}
\end{pmatrix}
,
\end{equation}
representado como en la siguiente imagen reescribiendo
\eqref{MicroscopicDielectric-Op} proyectada en los subespacios $a$ y
$f$,
\begin{equation}
  \label{CamposCompletos}
  \begin{pmatrix}
    {\bf D}_{a} \\
    {\bf D}_{f}
  \end{pmatrix}
  =
  \begin{pmatrix}
    \hat{\epsilon}_{aa} & \hat{\epsilon}_{af} \\
    \hat{\epsilon}_{fa} & \hat{\epsilon}_{ff}
  \end{pmatrix}
  \begin{pmatrix}
    {\bf E}_{a} \\
    {\bf E}_{f}
  \end{pmatrix}
\end{equation}
donde se ha definido
\begin{equation}
  \begin{array}{cc}
    \hat{O}_{\alpha \beta}\equiv \hat{P}_{\alpha}\hat{O}\hat{P}_{\beta}, & \alpha,\beta =a,f,
  \end{array}
\end{equation}
para cualquier operador.

Para calcular la respuesta macroscópica del sistema desacoplamos
${\bf D}_{a}$ y ${\bf E}_{f}$ de la ecuación de campos completos
\eqref{CamposCompletos}. Usando las ecuaciones de Maxwell para
los campos microscópicos se encuentra una relación entre ${\bf E}_{a}$
y ${\bf E}_{f}$,
\begin{equation}
  \begin{split}
      \nabla \times \nabla \times {\bf E} = \frac{4\pi
        i\omega}{c^{2}}{\bf j}_{ex} + \frac{\omega^{2}}{c^{2}}{\bf D}
      \\ \left[ \hat{\epsilon}_{ff} - \frac{c^{2}}{\omega^{2}}(\nabla
        \times \nabla \times)_{ff}\right] {\bf E}_{f} = \frac{4\pi}{i
        \omega }{\bf j}_{f}^{ex}-\hat{\epsilon}_{fa}{\bf E}_{a}
  \end{split}
\end{equation}
asumiendo que ${\bf J}^{ex}_{f} = 0 $, sustituimos ${\bf E}_{f}$ en la
ecuación \eqref{CamposCompletos},
\begin{equation}
  {\bf
    D}_{a}=\left(\hat{\epsilon}_{aa}-(\hat{\epsilon}_{af}(\hat{\epsilon}_{ff}-\frac{c^{2}}{\omega^{2}}(\nabla
  \times \nabla)_{ff})^{-1}\hat{\epsilon}_{fa}\right){\bf E}_{a},
  \label{MacroscopicRelation}
\end{equation}
comparando con la ecuación de campos macroscópicos obtenemos que
\begin{equation}
  \hat{\epsilon}_{M}=\hat{\epsilon}_{aa}-\hat{\epsilon}_{af}\left[(\hat{\epsilon}_{ff}-\frac{c^{2}}{\omega^{2}}(\nabla
    \times \nabla)_{ff})\right]^{-1}\hat{\epsilon}_{fa},
  \label{DielectricFunction}
\end{equation}
la respuesta macroscópica de un sistema está dada por el promedio de
su respuesta microscópica, más una corrección debida al acoplamiento
entre las componentes fluctuante y promedio de los campos, éste
resultado se presume exacto.

Para analizar el resultado de \eqref{DielectricFunction} nos
restringimos a una longitud de escala característica de las
fluctuaciones mucho menor a la longitud de onda. Convenitntemente se
definen los operadores de proyección longitudinal
\begin{equation}
  \hat{P}^{L} = \hat{\nabla} \hat{\nabla}^{-2} \hat{\nabla}\times,
\end{equation}
y transversal
\begin{equation}
  \hat{P}^{T}=-\hat{\nabla}\times \hat{\nabla}^{-2} \hat{\nabla}\times,
\end{equation}
tales que extraen las partes irrotacional ${\bf F}^{L}=\hat{P}^{L}{\bf
  F} $ y solenoidal ${\bf F}^{T}=\hat{P}^{T}{\bf F}$ de una campo
vectorial arbitrario. Estos operadores cumplen las siguiente
propiedades
\begin{equation}
  \begin{array}{cc}
    \hat{P}^{T}\hat{P}^{T}=\hat{P}^{T}, & \hat{P}^{L}\hat{P}^{L}=\hat{P}^{L},\\
    \hat{P}^{T}\hat{P}^{L}=\hat{P}^{L}\hat{P}^{T}=0, & \hat{P}^{T}+\hat{P}^{L}=\hat{1},
  \end{array}
\end{equation}
y communtan con $\hat{P}_{f}$ y $\hat{P}_{a}$.

Al proyectar $\hat{\mathcal{W}}^{-1}=
\left[(\hat{\epsilon}_{ff}-\frac{c^{2}}{\omega^{2}}(\nabla \times
  \nabla \times)_{ff})\right]^{-1}$ en los subespacios longitudina y
transversal
\[\hat{\mathcal{W}}^{-1}=\left[
    \begin{array}{cc}
    \mathcal{W}_{LL} &  \mathcal{W}_{LT} \\
    \mathcal{W}_{TL} &  \mathcal{W}_{TT}
    \end{array}
    \right]^{-1}, \] y reescribiendo el termino de operadores nabla
como, $(\nabla \times \nabla \times)_{ff} = \nabla
^{2}\hat{P}^{T}\hat{P}_{f}$, tenemos
\begin{equation}
  \label{MatrizdeProyeccionesdeW}
  \left[(\hat{\epsilon}_{ff}-\frac{c^{2}}{\omega^{2}}(\nabla \times \nabla \times)_{ff})\right]^{-1} = \left[
  \begin{array}{cc}
    \hat{\epsilon}_{ff}^{LL} & \hat{\epsilon}_{ff}^{LT}
    \\ \hat{\epsilon}_{ff}^{TL} & \hat{\epsilon}_{ff}^{TT}+
    \frac{c^{2}}{\omega^{2}}\nabla ^{2}\hat{P}^{T}\hat{P}_{f}
  \end{array}
  \right]^{-1}.
\end{equation}

La manera en que obtenemos las componentes de la matriz es la
siguiente, definimos una matriz y su inversa

\[ \hat{M} = \left(
\begin{array}{cc}
  \hat{A} &  \hat{B} \\
   \hat{C} &  \hat{D}
\end{array}\right),
\]
\[ \hat{m} =\left(
\begin{array}{cc}
   \hat{a} &  \hat{b} \\
   \hat{c} &  \hat{d}
\end{array}\right) = \hat{M}^{-1},
\]
la cuales deben cumplir con las ecuaciones
\begin{equation}
  \begin{split}
    \hat{A} \hat{a}+ \hat{B} \hat{c} = 1, \\
    \hat{A} \hat{b}+ \hat{B} \hat{d} = 0, \\
    \hat{C} \hat{a}+ \hat{D} \hat{c} = 0, \\
    \hat{C} \hat{b}+ \hat{D} \hat{d} = 1,
  \end{split}
\end{equation}
de las cuales uno despeja
\begin{equation}
  \label{eqsalgebraicas}
  \begin{split}
    \hat{a} = \hat{A}^{-1}(1-\hat{B}\hat{D}^{-1}\hat{C}\hat{A}^{-1})^{-1}, \\
    \hat{b} = \hat{C}^{-1}(1-\hat{D}\hat{B}^{-1}\hat{A}\hat{C}^{-1})^{-1}, \\
    \hat{c} = - \hat{D}^{-1}\hat{C}\hat{A}^{-1}(1-\hat{B}\hat{D}^{-1}\hat{C}\hat{A}^{-1})^{-1}, \\
    \hat{d} = - \hat{B}^{-1}\hat{A} \hat{C}^{-1}(1-\hat{D}\hat{B}^{-1}\hat{A}\hat{C}^{-1})^{-1}.
  \end{split}
\end{equation}

Para usar este desarrollo definimos $\hat{M} = \hat{\mathcal{W}}$ por
lo que los elementos respectivos son
\[
\begin{split}
  \hat{A} = \hat{\epsilon}_{ff}^{LL}, \\
  \hat{B} = \hat{\epsilon}_{ff}^{LT}, \\
  \hat{C} = \hat{\epsilon}_{ff}^{TL}, \\
  \hat{D} = \hat{\epsilon}_{ff}^{TT}+\frac{c^{2}}{\omega^{2}}\nabla ^{2}\hat{P}^{T}\hat{P}_{f},
\end{split}
\]
y asumiendo que $\lambda^{2}/l^{2} >> ||\hat{\epsilon}||$ donde $l$ es
la longitud de escala de las fluctuaciones, y que
$||(\omega^{2}/c^{2})\hat{\nabla}^{-2}\hat{P}_{f}||\approx
l^{2}/\lambda ^{2}$, entonces el segundo termino del elemento
$\hat{D}$ es mucho mayor que el primero. Para hacer el cálculo más
explícito lo reescribimos como
\[ \frac{c^{2}}{\omega^{2}}\nabla ^{2}\hat{P}^{T}\hat{P}_{f}(1+\frac{\omega^{2}}{c^{2}}\nabla ^{-2}\hat{P}^{T}\hat{P}_{f}\hat{\epsilon}_{ff}^{TT}) \]
y al sustituir en las ecuaciones \eqref{eqsalgebraicas}, obtenemos
\begin{equation}
  \begin{split}
    &\left[(\hat{\epsilon}_{ff}- \frac{c^{2}}{\omega^{2}}(\nabla \times
      \nabla \times)_{ff})\right]^{-1}  =
    \left[ \begin{array}{cc}
        (\hat{\epsilon}_{ff}^{LL})^{-1} & 0 \\ 0 & 0 \\
    \end{array}
      \right] \\ & + \frac{\omega^{2}}{c^{2}}\left[ \begin{array}{cc}
        (\hat{\epsilon}_{ff}^{LL})^{-1}(\hat{\epsilon}_{ff}^{LT})\hat{\nabla}^{-2}(\hat{\epsilon}_{ff}^{TL})(\hat{\epsilon}_{ff}^{LL})^{-1}
        &
        -(\hat{\epsilon}_{ff}^{LL})^{-1}(\hat{\epsilon}_{ff}^{LT})\hat{\nabla}^{-2}
        \\ -\hat{\nabla}^{-2}(\hat{\epsilon}_{ff}^{TL})(\hat{\epsilon}_{ff}^{LL})^{-1}
        & \nabla^{-2}\hat{P}^{T}\hat{P}_{f}
      \end{array} \right] + ...,
    \end{split}
\end{equation}
en los casos en que $l << \lambda $, solo nos quedamos con el primer
termino de la expansión y sustituyéndolo en \eqref{DielectricFunction}
la reescribimos como
\begin{equation}
  \hat{\epsilon}_{M}=\hat{\epsilon}_{aa}-\hat{\epsilon}_{af}(\hat{\epsilon}_{ff}^{LL})^{-1}\hat{\epsilon}_{fa}.
\end{equation}



\section{Implementación}
\section{Resultados}

\begin{thebibliography}{0000}
\bibitem{Metamorphose} Virtual Institute for Artificial
  Electromagnetic Materials and
  Meta-materials. \url{https://www.metamorphose-vi.org/}
\bibitem{IntroductiontoMetamaterialsandNanophotonics} An introduction
  to metamaterials and nanophotonics / Constantin Simovski and Sergei
  Tretyakov, Department of Electronics and Nanoengineering, School of
  Electrical Engineering, Aalto University,
  Finland. \url{http://www.cambridge.org/9781108492645}
\bibitem{ElectromagneticResponseofSystemwithSpatialFluctuations}
  Electromagnetic response of systems with spatial
  fluctuations. I. General formalism. W. Luis Mochán and Rubén
  G. Barrera. Phys. Rev. B 32, 4984 – Published 15 October 1985
\bibitem{Bulkplasmon} Mochán W.L., Plasmons. In: Saleem Hashmi
  (editor-in-chief), Reference Module in Materials Science and
  Materials. Engineering. Oxford: Elsevier; 2016. pp. 1-13. ISBN:
  978-0-12-803581-8. Copyright © 2016 Elsevier Inc. unless otherwise
  stated. All rights reserved.
\bibitem{LycurgusInvestigation} Barber DJ, Freestone IC (1990) An
  investigation of the origin of the color of the Lycurgus cup by
  analytical transmission electron-microscopy. Archaeometry 32:33–45
\bibitem{Fuchs}Theory of the optical properties of ionic crystal
  cubes, Physical Review B {\bf 11}, 1732--1740, (1975),
  \href{http://link.aps.org/doi/10.1103/PhysRevB.11.1732}{doi:10.1103/PhysRevB.11.1732}

\end{thebibliography}

\end{document}
